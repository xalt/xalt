\documentclass{beamer}

% You can also use a 16:9 aspect ratio:
%\documentclass[aspectratio=169]{beamer}
\usetheme{TACC16}
\usepackage{graphicx}

% It's possible to move the footer to the right:
%\usetheme[rightfooter]{TACC16}


% Detailed knowledge of application workload characteristics can
% optimize performance of current and future systems. This may sound
% daunting, with many HPC data centers hosting over 2,000 users running
% thousands of applications and millions of jobs per month.  XALT is an
% open source tool developed at the Texas Advanced Computing Center
% (TACC) that collects system usage information to quantitatively report
% how users are using your system. This session will explore the
% benefits of detailed application workload profiling and how the XALT
% tool has helped leading supercomputing sites unlock the power of their
% application usage data.

%% page 
%\begin{frame}{}
%  \begin{itemize}
%    \item
%  \end{itemize}
%\end{frame}
%
%% page 
%\begin{frame}[fragile]
%    \frametitle{}
% {\tiny
%    \begin{semiverbatim}
%    \end{semiverbatim}
%}
%  \begin{itemize}
%    \item
%  \end{itemize}
%
%\end{frame}



\begin{document}
\title[XALT]{Json ingestion improvements}
\author{Robert McLay}
\date{May 19, 2022}

% page 1
\frame{\titlepage}

% page 2
\begin{frame}{XALT: Outline}
  \center{\includegraphics[width=.9\textwidth]{XALT_Sticker.png}}
  \begin{itemize}
    \item Ingestion Definition
    \item Questions about how site install XALT and Ingestion
    \item Execution Filtering
    \item New Package Filtering
    \item Debugging *.json file ingestion
    \item Tracking Down when Json Ingestion is slow
    \item Debugging *.json syslog ingestion
  \end{itemize}
\end{frame}

% page 3
\begin{frame}{Ingestion Definition}
  \begin{itemize}
    \item XALT generates Json records (either file or syslog or ...)
    \item Ingestion is the phase where Json records are added to the
      MySQL DB.
  \end{itemize}
\end{frame}

% page 4
\begin{frame}{How sites install XALT for users?}
  \begin{itemize}
    \item XALT is installed locally on each node? (TACC does this)
    \item XALT is in a shared global location?
    \item Other ways?
  \end{itemize}
\end{frame}

% page 5
\begin{frame}{How sites install XALT for Ingestion?}
  \begin{itemize}
    \item TACC uses a VM to store the syslog transmission style
    \item We have also used a shared global file location to ingest
      json records
    \item Other ways?
  \end{itemize}
\end{frame}


% page 6
\begin{frame}{Dealing with different types of XALT installation}
  \begin{itemize}
    \item This talk assumes that installing XALT for users is more
      difficult
    \item At TACC, XALT for users can only be updated on maintenance days 
    \item This may get easier in the future at TACC with rolling instations
    \item Where as changes to ingestion are easy.
    \item We only change the XALT installation on the VM.
    \item Is this different for your site?
  \end{itemize}
\end{frame}

% page 7
\begin{frame}{Executable Pre-ingestion filtering}
  \begin{itemize}
    \item Because some sites have trouble updating XALT for users
    \item There was a request to allow filtering of *.json records 
    \item These are transmitted records but not yet ingested
  \end{itemize}
\end{frame}

% page 8
\begin{frame}[fragile]
    \frametitle{A new python array to filter executables}
 {\small
    \begin{semiverbatim}
pre_ingest_patterns = [
#   percent   path pattern
    [0.0,     r'.*\textbackslash{}/foobar'],
    [0.01,    r'.*\textbackslash{}/BAZ'],
]
    \end{semiverbatim}
 }
  \begin{itemize}
    \item This python array is converted to a flex routine
    \item This flex code is converted to C code and compiled into a
      shared library (libpreIngest.so)
    \item Python allows for standard libraries be connected to python programs
  \end{itemize}
\end{frame}

% page 9
\begin{frame}[fragile]
    \frametitle{How filtering is shown in xalt\_configuration\_report}
 {\tiny
    \begin{semiverbatim}
\% xalt\_configuration\_report
...        
*----------------------*
 Array: ingestPatternA
*----------------------*
================== /path/to/site/config.py ==================
   0: 0.0000 => .*\textbackslash{}/foobar
   1: 0.0100 => .*\textbackslash{}/BAZ
================== src/tmpl/xalt\_config.py ==================
   2: 1.0000 => .*
    \end{semiverbatim}
}
  \begin{itemize}
    \item Note that src/tmpl/xalt\_config.py  provides the default
      patterns.
    \item So you don't have to.
  \end{itemize}

\end{frame}

% page 10
\begin{frame}[fragile]
    \frametitle{Connecting C shared libraries w/ Python}
 {\tiny
    \begin{semiverbatim}
from   ctypes import *   # used to interact with C shared libraries

libpath      = os.path.join(dirNm, "../lib64/libpreIngest.so")
libpreIngest = CDLL(libpath)
pre\_ingest\_filter = libpreIngest.pre\_ingest\_filter
pre\_ingest\_filter.argtypes = [c\_char\_p]
pre\_ingest\_filter.restype  = c\_double
...
exec\_prob = pre\_ingest\_filter(exec\_path.encode())
prob      = random.random() # 0 < prob <= 1.0
if (prob <= exec\_prob):
   # ingest

    \end{semiverbatim}
}
  \begin{itemize}
    \item The ctypes python package provide the magic.
    \item Once the boilerplate code is provided, it is just one line
      to find the probability to keep or not.
    \item This is very fast filtering 
  \end{itemize}
\end{frame}

% page 11
\begin{frame}[fragile]
    \frametitle{Pkg Filtering added to Ingestion}
  \begin{itemize}
    \item XALT 2.10.37 added Pkg filtering to xalt\_record\_pkg prgm
    \item It also added the same filtering to Ingestion 
    \item It uses the same ctypes package to integrate it with python.
    \item This means that you can filter package w/o re-installing
      XALT everywhere.
  \end{itemize}
\end{frame}

% page 12
\begin{frame}[fragile]
    \frametitle{Protecting XALT against endless loops from site
      Config.py files}
 {\tiny
    \begin{semiverbatim}
# A site had in their site Config.py: 
path_patterns = [
  ['KEEP',  r'\textbackslash{}/opt\textbackslash{}/envHPC\textbackslash{}/.*'],
  [ 'SKIP',  r'.*'],
  ]
    \end{semiverbatim}
}
  \begin{itemize}
    \item where XALT was stored in /opt/envHPC/xalt/...
    \item XALT must protect sites from tracking all xalt programs
    \item Otherwise you can get an endless loop.
  \end{itemize}
\end{frame}

% page 13
\begin{frame}[fragile]
    \frametitle{Preventing Endless XALT loops}
  \begin{itemize}
    \item XALT has a system config file: src/tmpl/xalt\_config.py
    \item Site has a Config.py file
    \item This files control how XALT filters
    \item The system file has head\_path\_patterns
    \item This forces XALT to skip all XALT executabls
    \item Also skip the unix system logger command which write to
      syslog.
    \item XALT uses logger in testing.
  \end{itemize}
\end{frame}

% page 14
\begin{frame}[fragile]
    \frametitle{Configuration Report}
 {\tiny
    \begin{semiverbatim}
*----------------------*
 Array: pathPatternA
*----------------------*
====== src/tmpl/xalt\_config.py =====
   1: SKIP => .*\textbackslash{}/logger
   6: SKIP => .*\textbackslash{}/xalt\_syshost
   7: SKIP => .*\textbackslash{}/xalt\_record\_pkg
======= Config.py ============
...
====== src/tmpl/xalt\_config.py =====
  21: KEEP => .*

    \end{semiverbatim}
}
  \begin{itemize}
    \item Abbreviated patterns from xalt\_configuration\_report
  \end{itemize}
\end{frame}

% page 15
\begin{frame}{Debugging Json ingestion}
  \begin{itemize}
    \item Issue \#46 shows a detailed discussion where a site had
      trouble ingesting.
    \item It wasn't clear where the slowdown was happening 
    \item xalt\_file\_to\_db.py and xalt\_syslog\_to\_db.py got the -D
      option 
    \item This adds debug printing for the internal steps
  \end{itemize}
\end{frame}

% page 16
\begin{frame}[fragile]
    \frametitle{XALT searches first for link.*.json files}
 {\tiny
    \begin{semiverbatim}
link: /home/user/.xalt.d/link.rios.*.json
  --> Trying to open file
  --> Trying to load json
  --> Sending record to xalt.link_to_db()
  --> Trying to connect to database
  --> Starting TRANSACTION
  --> Searching for build_uuid in db
  --> Trying to insert link record into db
  --> Success: link recorded
  --> Trying to insert objects into db
  --> Trying to insert functions into db
  --> Done
    \end{semiverbatim}
}
\end{frame}


% page 17
\begin{frame}[fragile]
    \frametitle{XALT searches for run.*.json files next}
 {\tiny
    \begin{semiverbatim}
run: /home/user/.xalt.d/run.rios.*.json
  --> Trying to open file
  --> Trying to load json
  --> Sending record to xalt.run_to_db()
  --> Trying to connect to database
  --> Starting TRANSACTION
  --> Searching for run_uuid in db
  --> Trying to insert run record into db
  --> Success: stored full xalt_run record
  --> Trying to insert objects into db
  --> Trying to insert env vars into db
  --> Done
    \end{semiverbatim}
}

\end{frame}

% page 18
\begin{frame}[fragile]
    \frametitle{Finally XALT searches for pkg.*.json}
 {\tiny
    \begin{semiverbatim}
  --> Found 10 pkg.*.json files

  --> Success: pkg entry "R:bar" stored
  --> Success: pkg entry "R:foo" stored
  --> Success: pkg entry "R:acme" stored
  --> Failed to record: pkgFilter blocks "R:base" 
  --> Success: pkg entry "python:json" stored
  --> Success: pkg entry "python:linecache" stored
  --> Success: pkg entry "python:struct" stored
  --> Success: pkg entry "python:base64" stored
  --> Success: pkg entry "python:codecs" stored
    \end{semiverbatim}
}
\end{frame}

% page 19
\begin{frame}{Debugging Results}
  \begin{itemize}
    \item Site used $\sim$user/.xalt.d
    \item BTW: This has a race condition 
    \item The $\sim$/.xalt.d directory has to exist first
    \item Walking a parallel file system is slow when looking
      $\sim$/.xalt.d directories.
    \item I encourage the site to switch to a global shared location
  \end{itemize}
\end{frame}

% page 20
\begin{frame}[fragile]
    \frametitle{Similarly for xalt\_syslog\_to\_db.py}
 {\tiny
    \begin{semiverbatim}
  --> Trying to connect to database
  --> Starting TRANSACTION
  --> Searching for build_uuid in db
  --> Trying to insert link record into db
  --> Success: link recorded
  --> Trying to insert objects into db
  --> Trying to insert functions into db
  --> Done

  --> Trying to connect to database
  --> Starting TRANSACTION
  --> Searching for run_uuid in db
  --> Trying to insert run record into db
  --> Success: stored full xalt_run record
  --> Trying to insert objects into db
  --> Trying to insert env vars into db
  --> Done

  --> Success: pkg entry "python:token" stored
  --> Success: pkg entry "python:tokenize" stored
  --> Success: pkg entry "python:linecache" stored
    \end{semiverbatim}
}
  \begin{itemize}
    \item  But there there are no file names given and the link, run and pkg
  records are mixed together
  \end{itemize}

\end{frame}


% page 21
\begin{frame}{Conclusions}
  \begin{itemize}
    \item Debugging ingestion is not practical for thousands for
      records.
    \item But it is useful.
    \item Next Meeting June 16 10:00am U.S. Central (15:00 UTC)
  \end{itemize}
\end{frame}


% page 22
\begin{frame}{Future Topics?}
  \begin{itemize}
    \item ???
    \item Others?
  \end{itemize}
\end{frame}
%

%\input{./themes/license}

\end{document}
