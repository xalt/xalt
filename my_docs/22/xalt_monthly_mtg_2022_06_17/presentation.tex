\documentclass{beamer}

% You can also use a 16:9 aspect ratio:
%\documentclass[aspectratio=169]{beamer}
\usetheme{TACC16}
\usepackage{graphicx}

% It's possible to move the footer to the right:
%\usetheme[rightfooter]{TACC16}


% Detailed knowledge of application workload characteristics can
% optimize performance of current and future systems. This may sound
% daunting, with many HPC data centers hosting over 2,000 users running
% thousands of applications and millions of jobs per month.  XALT is an
% open source tool developed at the Texas Advanced Computing Center
% (TACC) that collects system usage information to quantitatively report
% how users are using your system. This session will explore the
% benefits of detailed application workload profiling and how the XALT
% tool has helped leading supercomputing sites unlock the power of their
% application usage data.

%% page 
%\begin{frame}{}
%  \begin{itemize}
%    \item
%  \end{itemize}
%\end{frame}
%
%% page 
%\begin{frame}[fragile]
%    \frametitle{}
% {\tiny
%    \begin{semiverbatim}
%    \end{semiverbatim}
%}
%  \begin{itemize}
%    \item
%  \end{itemize}
%
%\end{frame}



\begin{document}
\title[XALT]{XALT Changes to support Sampling and why Signaling didn't
save us}
\author{Robert McLay}
\date{May 19, 2022}

% page 1
\frame{\titlepage}

% page 2
\begin{frame}{XALT: Outline}
  \center{\includegraphics[width=.9\textwidth]{XALT_Sticker.png}}
  \begin{itemize}
    \item XALT 2 Can track everything 
    \item This is bad unless you want to throw a Supercomputer to
      collect data.
    \item XALT has several tricks to manage the firehose of data
    \item 1st Level Defense: Path Filtering
    \item 2nd Level Defense: Sampling
    \item 3rd Level Defense: Signals and why they don't work
  \end{itemize}
\end{frame}

% page 3
\begin{frame}{History}
  \begin{itemize}
    \item XALT 1 only tracked MPI programs by modifying mpirun, ibrun etc.
    \item XALT 2 can track everything via libelf trick
    \item See xalt.readthedocs.io and libelf\_trick in source repo on
      how this works.
    \item 1st line of defense: path filtering
  \end{itemize}
\end{frame}

% page 4
\begin{frame}[fragile]
    \frametitle{Path Filtering}
  \begin{itemize}
    \item XALT allows site to have paths to KEEP or SKIP
    \item Your site \texttt{Config.py} has a section like this:
    \item This allows sites to pick some program like \texttt{sed,
        perl} to track and ignore others.
  \end{itemize}
 {\small
    \begin{semiverbatim}
path\_patterns = [
    ['PKGS',  r'.*\\/R'],
    ['PKGS',  r'.*\\/MATLAB'],
    ['PKGS',  r'.*\\/python[0-9.]*'],
    ['KEEP',  r'^\\/usr\\/bin\\/sed'],
    ['KEEP',  r'^\\/usr\\/bin\\/perl'],
    ['SKIP',  r'^\\/usr\\/.*'],
    ['SKIP',  r'^\\/sbin\\/.*'],
    ['SKIP',  r'^\\/bin\\/.*'],
    \end{semiverbatim}
}

\end{frame}

% page 5
\begin{frame}{}
    But that is not sufficient!!
\end{frame}

% page 6
\begin{frame}{Not Sufficent!}
  \begin{itemize}
    \item A two hour image processing generated millions of json records. 
    \item It took over 4 days to process this one day's result.
    \item Obviously, this is not sustainable.
  \end{itemize}
\end{frame}


% page 7
\begin{frame}{Not Sufficient, part 2}
  \begin{itemize}
    \item I thought that MPI programs were safe from thru-put
      computing, But...
    \item At least one group is using short time 4 task MPI prgms to
      train a neural network.
    \item So both MPI and NON-MPI executions must be sampled.
  \end{itemize}
\end{frame}

% page 8
\begin{frame}[fragile]
    \frametitle{Non-MPI Programming Sampling}
  \begin{itemize}
    \item As I have mention before
    \item Scalar (Non-MPI) execution can be configured to be sampled
    \item TACC's Scalar rules are below.
    \item Longer execution increases change of being tracked
  \end{itemize}
 {\small
    \begin{semiverbatim}
interval\_array = [
    [    0.0,              0.0001 ],
    [ 1800.0,              0.01   ],
    [ 7200.0,              1.0    ],
    [ sys.float\_info.max,  1.0    ]
]
    \end{semiverbatim}
}

\end{frame}

% page 9
\begin{frame}[fragile]
    \frametitle{MPI Programming Sampling}
  \begin{itemize}
    \item TACC's MPI rules are below.
    \item Longer execution increases change of being tracked
  \end{itemize}
 {\small
    \begin{semiverbatim}
mpi\_interval\_array = [
    [    0.0,              0.0001 ],
    [  900.0,              0.01   ],
    [ 1800.0,              1.0    ],
    [ sys.float\_info.max,  1.0    ]
]
    \end{semiverbatim}
}
\end{frame}

% page 10
\begin{frame}{Consequences of Sampling}
  \begin{itemize}
    \item Before Sampling XALT generated both a start and end json
      record.
    \item XALT only produces records on the ``wire'' that are meant to
      be saved.
    \item $\Rightarrow$ no start record for sampled data
    \item Can't know how long something will run at the beginning
    \item So only an end record is produced
    \item But this means that the program must end normally.
    \item Means that segfault etc runs will be ignored
  \end{itemize}
\end{frame}

% page 11
\begin{frame}{Consequences of Sampling: Package Records}
  \begin{itemize}
    \item A package record can be generated for python, R, MATLAB at
      anytime.
    \item A package record needs an execution record to ``hang-on''
      to.
    \item Instead of forcing all python and R execution to produce a
      start record.
    \item ALL package json records are written to /dev/shm
    \item If chosen then json records are copied to ``wire''
    \item If not then deleted at end of execution 
    \item Or left to hang if python job doesn't complete.
  \end{itemize}
\end{frame}

% page 12
\begin{frame}{What about long running MPI simulations?}
  \begin{itemize}
    \item Some Large MPI execution never terminate
    \item Long running simulations like Weather or other calculations 
    \item These run as many timesteps as they can in 24 or 48 hours
      job window.
    \item They restart from the last timestep or similar results file.
  \end{itemize}
\end{frame}

% page 13
\begin{frame}{Long running MPI Simulations (II)}
  \begin{itemize}
    \item XALT has a configuration option \texttt{MPI\_ALWAYS\_RECORD} 
    \item Typically set to 128 MPI Tasks (Site configurable)
    \item Hopefully no-one is training Neural Networks with 128 or
      more Tasks.
    \item This means that XALT does NOT sample any executions 128
      tasks or more.
    \item Sites can use the endtime from ``SLURM'' accounting to set
      the endtime that doesn't exist. 
  \end{itemize}
\end{frame}

% page 14
\begin{frame}{What about using Signals to capture scalar execution?}
  \begin{itemize}
    \item XALT has support for capturing end records for segfault etc
      jobs.
    \item It is off by default now.
    \item This works well for almost all executions
    \item But occasionally it doesn't with some python programs
    \item And I don't understand why.
    \item XALT sets the signal handler first
    \item This allows a user program to overwrite XALT's handler
    \item But this sometime causes problems.
    \item But since XALT has to sample scalar jobs anyway 
    \item This is not problem worth solving
  \end{itemize}
\end{frame}

% page 15
\begin{frame}{What about using Signals to capture Long running MPI execution?}
  \begin{itemize}
    \item SLURM sends a signal to a job that it is about to timeout.
    \item This works with scalar executions
    \item This could be used by XALT to write the end json record
    \item If this worked, XALT could stop writing start json records 
    \item But it doesn't work the way I want it to.
  \end{itemize}
\end{frame}

% page 16
\begin{frame}{Signals and MPI executions (II)}
  \begin{itemize}
    \item However the controlling MPI executable doesn't pass this
      signal on to the user program.
    \item Our resident MPI expert Amit Ruhela tried to get this to
      work with Intel MPI and Mvapich2 but couldn't
    \item This means that large MPI execution $>$ 127 tasks are not sampled.
  \end{itemize}
\end{frame}

% page 17
\begin{frame}{Signals and MPI executions (III)}
  \begin{itemize}
    \item I not sure that this would always work anyway
    \item The timeout signal would have to be captured by task 0.
    \item Remember XALT does nothing on all other tasks.
  \end{itemize}
\end{frame}


% page 18
\begin{frame}{Conclusions}
  \begin{itemize}
    \item Sampling saved XALT from the firehose of data from thru-put computing
    \item Sampling works for Scalar and Small MPI executions
    \item Signals won't save us.
  \end{itemize}
\end{frame}


% page 19
\begin{frame}{Future Topics?}
  \begin{itemize}
    \item How XALT ``hijack'' the link process to watermark user
      programs.
    \item This hijacking is not required for execution tracking
    \item Next Meeting will be on July 22, 2022 at 10:00 am
      U.S. Central (15:00 UTC)
    \item This talk will be given by my colleague: Amit Ruhela.
  \end{itemize}
\end{frame}

%\input{./themes/license}

\end{document}
