\documentclass{beamer}

% You can also use a 16:9 aspect ratio:
%\documentclass[aspectratio=169]{beamer}
\usetheme{TACC16}
\usepackage{graphicx}

% It's possible to move the footer to the right:
%\usetheme[rightfooter]{TACC16}


% Detailed knowledge of application workload characteristics can
% optimize performance of current and future systems. This may sound
% daunting, with many HPC data centers hosting over 2,000 users running
% thousands of applications and millions of jobs per month.  XALT is an
% open source tool developed at the Texas Advanced Computing Center
% (TACC) that collects system usage information to quantitatively report
% how users are using your system. This session will explore the
% benefits of detailed application workload profiling and how the XALT
% tool has helped leading supercomputing sites unlock the power of their
% application usage data. 


\begin{document}
\title[XALT]{Tracking imports for Python, R}
\author{Robert McLay} 
\date{Feb. 17, 2022} 

% page 1
\frame{\titlepage} 

% page 2
\begin{frame}{XALT: Outline}
  \center{\includegraphics[width=.9\textwidth]{XALT_Sticker.png}}
  \begin{itemize}
    \item XALT can track executable that are run
    \item Also the shared libraries
    \item Can we track imports for Python and R?
    \item How could we do it?
  \end{itemize}
\end{frame}

% page 3
\begin{frame}{How can we track imported packages?}
  \begin{itemize}
    \item It will require special code unique for each tool
    \item Somehow we have to ``insert'' code into the import process
    \item This is typically accomplished by using some Hook provide by
      the tools' developers
  \end{itemize}
\end{frame}

% page 3
\begin{frame}{Tracking imports of packages for R}
  \begin{itemize}
    \item It started with R.
    \item James McComb \& Michael Scott from IU developed the R part
    \item They wrote code that intercepts the import action.
    \item XALT provides a program to take that data: xalt\_record\_pkg
    \item All packages tracker use this program to collect the data.
  \end{itemize}
\end{frame}

% page 4
\begin{frame}{XALT Prerequisites}
  \begin{itemize}
    \item A \texttt{path\_pattern} in a sites' Config.py.
    \item \texttt{ ['PKGS',  r'.*\/R'],}
    \item \texttt{ ['PKGS',  r'.*\/python[0-9.]*'],}
  \end{itemize}
\end{frame}

% page 5
\begin{frame}{XALT Env. Vars for PKGS}
  \begin{itemize}
    \item Since the execution is happenning during XALT Tracking
    \item The following environment variables are defined
      \begin{itemize}
        \item XALT\_DIR: The root directory of where xalt\_record\_pkg
        \item XALT\_RUN\_UUID: The run uuid for the current R or
          Python program running
      \end{itemize}
    \item The R and Python Import hooks only collect data if
      XALT\_RUN\_UUID is defined
  \end{itemize}
\end{frame}

% page 6
\begin{frame}{\texttt{xalt\_record\_pkg} usage}
  \begin{itemize}
    \item The hook routine does the following
    \item Gets XALT\_DIR and XALT\_RUN\_UUID from env.
    \item Builds command path to \texttt{xalt\_record\_pkg} using
      XALT\_DIR
    \item The rest of the command line is:
      \begin{itemize}
        \item -u $<$run\_uuid$>$
        \item program $<$name$>$
        \item xalt_run_uuid $<$run\_uuid$>$
        \item package\_name $<$pkg_name$>$
        \item package\_version $<$pkg_version$>$
        \item package\_path $<$pkg_path$>$
      \end{itemize}
  \end{itemize}
\end{frame}

% page 7
\begin{frame}{\texttt{xalt\_record\_pkg} execution}
  \begin{itemize}
    \item \texttt{xalt\_record\_pkg} builds a json string w/ data
    \item Every import will generate a record
    \item Writes data to tmp dir under /dev/shm
    \item Why?
  \end{itemize}
\end{frame}

% page 8
\begin{frame}{Why write package import data to /dev/shm?}
  \begin{itemize}
    \item Speed
    \item Python and R can be sampled
    \item Data is only sent on the ``wire'' at end of program if sampled
    \item Delete data otherwise.
  \end{itemize}
\end{frame}


% page 20
\begin{frame}{Conclusions}
  \begin{itemize}
    \item We can extract useful data from XALT
    \item It is not quick on a VM.
    \item Would love to have the DB on SSD
    \item We use data to know what codes add to next benchmark.
  \end{itemize}
\end{frame}


% page 21
\begin{frame}{Future Topics?} 
  \begin{itemize}
    \item Package tracking
    \item Others?
  \end{itemize}
\end{frame}
%

%\input{./themes/license}

\end{document}
