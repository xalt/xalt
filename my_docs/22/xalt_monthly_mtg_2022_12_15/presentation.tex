\documentclass{beamer}

% You can also use a 16:9 aspect ratio:
%\documentclass[aspectratio=169]{beamer}
\usetheme{TACC16}
\usepackage{graphicx}

% It's possible to move the footer to the right:
%\usetheme[rightfooter]{TACC16}


% Detailed knowledge of application workload characteristics can
% optimize performance of current and future systems. This may sound
% daunting, with many HPC data centers hosting over 2,000 users running
% thousands of applications and millions of jobs per month.  XALT is an
% open source tool developed at the Texas Advanced Computing Center
% (TACC) that collects system usage information to quantitatively report
% how users are using your system. This session will explore the
% benefits of detailed application workload profiling and how the XALT
% tool has helped leading supercomputing sites unlock the power of their
% application usage data.

%% page 
%\begin{frame}{}
%  \begin{itemize}
%    \item
%  \end{itemize}
%\end{frame}
%
%% page 
%\begin{frame}[fragile]
%    \frametitle{}
% {\tiny
%    \begin{semiverbatim}
%    \end{semiverbatim}
%}
%  \begin{itemize}
%    \item
%  \end{itemize}
%
%\end{frame}



\begin{document}
\title[XALT]{R and Python Package Collection and Filtering}
\author{Robert McLay}
\date{December 15, 2022}

% page 1
\frame{\titlepage}

% page 2
\begin{frame}{XALT: Outline}
  \center{\includegraphics[width=.9\textwidth]{XALT_Sticker.png}}
  \begin{itemize}
    \item XALT can record package usage by R and Python (optionally Matlab)
    \item This talk centers on Python with a little on others
    \item Special code is required to use allow for tracking
  \end{itemize}
\end{frame}

% page 3
\begin{frame}{R tracking requires a special package to turn on tracking}
  \begin{itemize}
    \item Package developed at IU by James McCombs and Michael Scott
    \item I need to get one of them for docs on how to install it.
  \end{itemize}
\end{frame}

% page 4
\begin{frame}{Python tracking}
  \begin{itemize}
    \item I developed XALT's xalt\_sitecustomize.py (installed as sitecustomize.py)
    \item This was with help from Riccardo Murri (with an intro from
      Kenneth Hoste!)
    \item To track python imports add \$XALT\_DIR/site\_packages to
      \$PYTHONPATH
    \item This puts XALT's sitecustomize.py to ALWAYS read by
      every python execution
  \end{itemize}
\end{frame}

% page 5
\begin{frame}[fragile]
    \frametitle{XALT package tracking interface}
  \begin{itemize}
    \item All package tracking ``hooks'' calls the program
      xalt\_record\_pkg to interface with XALT
    \item It can be found in \$XALT\_DIR/libexec
    \item In Python <NAME> or <PATH> might be marked as the string
      '<unknown>'.
    \item No tracking if both are unknown.
  \end{itemize}
 {\small
    \begin{semiverbatim}
../xalt\_record\_pkg program          <PROGRAM> \textbackslash
                   xalt\_run\_uuid    <UUID>    \textbackslash
                   package\_name     <NAME>     \textbackslash
                   package\_version  <VERSION>  \textbackslash
                   package\_path     <PATH>
    \end{semiverbatim}
}
\end{frame}

% page 6
\begin{frame}{How XALT handles package records}
  \begin{itemize}
    \item Since XALT won't know if the PKG prog will be tracked.
    \item All PKG programs will produce records but!
    \item All pkg records written to /dev/shmem or /tmp
    \item Non sampled runs remove records at end
    \item Sampled are send on ``wire'' (i.e. saved to db) via your
      transport method
    \item This way pkg records are written on the ``wire'' only when 
      necessary
  \end{itemize}
\end{frame}

% page 7
\begin{frame}{Save every python/R package?}
  \begin{itemize}
    \item Probably not
    \item There are two kinds of optional filtering
    \item Python only (part of xalt\_sitecustomize.py)
    \item All package filtering, available when accepting records to
      write to DB
  \end{itemize}
\end{frame}

% page 8
\begin{frame}[fragile]
    \frametitle{The block python\_pkg\_pattern}
 {\tiny
    \begin{semiverbatim}
python_pkg_patterns = [
  { 'k_s' : 'SKIP', 'kind' : 'path', 'patt' : r"^[^/]"      },
  { 'k_s' : 'SKIP', 'kind' : 'name', 'patt' : r"^_"         },
]
    \end{semiverbatim}
}
  \begin{itemize}
    \item A full example can be found in Config/rtm\_config.py
  \end{itemize}

\end{frame}

% page 9
\begin{frame}[fragile]
    \frametitle{Here are some of the rules we use}
 {\tiny
    \begin{semiverbatim}
  path : r"^[^/]"       # SKIP all built-in packages
  name : r"^_"          # SKIP names that start with a underscore
  name : r".*\textbackslash."        # SKIP all names that are divided with periods: a.b
  path :  r".*/.local/" # KEEP all packages installed by users
    \end{semiverbatim}
}
  \begin{itemize}
    \item Since this is part of sitecustomize.py
    \item No records are even written to /dev/shmem
  \end{itemize}
\end{frame}


% page 10
\begin{frame}[fragile]
    \frametitle{All Package Filtering}
  \begin{itemize}
    \item Filtering can by XALTdb.py (when loading into DB).  
    \item It uses a flex routine to filter both Python and R programs
    \item This block of code goes in your Config.py
    \item I should have just provided this version first but ...
  \end{itemize}
 {\tiny
    \begin{semiverbatim}
pkg_patterns = [
  ["SKIP",  r'^R:name:stats'],         # SKIP the R pkg named stats
  ["SKIP",  r'^R:name:base'],          # SKIP the R pkg named base
  ["SKIP",  r'^R:name:methods'],       # SKIP the R pkg named methods
  ["SKIP",  r'^python:name:_.*'],      # SKIP all python name that start with an underscore
  ["SKIP",  r'^python:path:[^/].*'],   # SKIP all python built-in packages
  ["SKIP",  r'^python:path:\/home'],   # SKIP all python package in user locations
]
    \end{semiverbatim}
}

\end{frame}

% page 10
\begin{frame}{Suggested Rules for using either or both filter systems}
  \begin{itemize}
    \item Use the Python only filtering for a few rules
    \item As this filtering happens for every import
    \item Save complicated filtering for the total filtering system
  \end{itemize}
\end{frame}

% page 11
\begin{frame}{Filtering opportunity}
  \begin{itemize}
    \item Currently all filtering is one path or name at a time
    \item Since all pkgs for are a single job are available at end
    \item It could be possible to provide an array of records before
      transmitting
    \item I'm looking for feedback. 
  \end{itemize}
\end{frame}

% page 12
\begin{frame}{Future Topics?}
  \begin{itemize}
    \item I'm looking for Topics.
    \item Next Meeting will be on Jan. 19, 2023 at 10:00 am
      U.S. Central (16:00 UTC)
  \end{itemize}
\end{frame}

%\input{./themes/license}

\end{document}
