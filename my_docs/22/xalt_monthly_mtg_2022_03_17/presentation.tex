\documentclass{beamer}

% You can also use a 16:9 aspect ratio:
%\documentclass[aspectratio=169]{beamer}
\usetheme{TACC16}
\usepackage{graphicx}

% It's possible to move the footer to the right:
%\usetheme[rightfooter]{TACC16}


% Detailed knowledge of application workload characteristics can
% optimize performance of current and future systems. This may sound
% daunting, with many HPC data centers hosting over 2,000 users running
% thousands of applications and millions of jobs per month.  XALT is an
% open source tool developed at the Texas Advanced Computing Center
% (TACC) that collects system usage information to quantitatively report
% how users are using your system. This session will explore the
% benefits of detailed application workload profiling and how the XALT
% tool has helped leading supercomputing sites unlock the power of their
% application usage data.

%% page
%\begin{frame}{}
%  \begin{itemize}
%    \item
%  \end{itemize}
%\end{frame}
%
%% page
%\begin{frame}[fragile]
%    \frametitle{}
% {\tiny
%    \begin{semiverbatim}
%    \end{semiverbatim}
%}
%  \begin{itemize}
%    \item
%  \end{itemize}
%
%\end{frame}



\begin{document}
\title[XALT]{Protecting XALT from Users}
\author{Robert McLay}
\date{March 17, 2022}

% page 1
\frame{\titlepage}

% page 2
\begin{frame}{XALT: Outline}
  \center{\includegraphics[width=.9\textwidth]{XALT_Sticker.png}}
  \begin{itemize}
    \item XALT is linking with every program that runs on the system
    \item Users make mistakes
    \item Need to protect XALT from user mistakes
    \item Show three protection examples 
  \end{itemize}
\end{frame}

% page 3
\begin{frame}{Three examples of protection}
  \begin{itemize}
    \item User's Expecting allocated memory to be zero'd
    \item User's mixing Fortran routine with C library routines
    \item XALT expecting well managed memory heap.
  \end{itemize}
\end{frame}


% page 4
\begin{frame}[fragile]
    \frametitle{How XALT works}
 {\tiny
    \begin{semiverbatim}
#include <stdio.h>
void myinit(int argc, char **argv)
\{ printf("This is run before main()\n"); \}
void myfini()
\{ printf("This is run after main()\n"); \}

__attribute__((section(".init_array"))) __typeof__(myinit) *__init = myinit;
__attribute__((section(".fini_array"))) __typeof__(myfini) *__fini = myfini;
    \end{semiverbatim}
}
\end{frame}

% page 5
\begin{frame}[fragile]
    \frametitle{How XALT works (II)}
 {\tiny
    \begin{semiverbatim}
#include <stdio.h>
int main()
{
  printf("Hello World!\n");
  return 0;
}

    \end{semiverbatim}
}
\end{frame}


% page 6
\begin{frame}[fragile]
    \frametitle{How XALT works (II)}}
 {\tiny
    \begin{semiverbatim}
\$ ./try; echo

Hello World!

\$ LD_PRELOAD=./libxalt.so  ./try  ; echo
This is run before main()
Hello World!
This is run after main()
    \end{semiverbatim}
}
\end{frame}


% page 4
\begin{frame}{How XALT works }
  \begin{itemize}
    \item It started with R.
    \item James McComb \& Michael Scott from IU developed the R part
    \item They wrote code that intercepts the import action.
    \item XALT provides a program to take that data: xalt\_record\_pkg
    \item All packages tracker use this program to collect the data.
  \end{itemize}
\end{frame}

% page 4
\begin{frame}{XALT Prerequisites}
  \begin{itemize}
    \item A \texttt{path\_pattern} in a sites' Config.py.
    \item \texttt{ ['PKGS',  r'.*\/R'],}
    \item \texttt{ ['PKGS',  r'.*\/python[0-9.]*'],}
  \end{itemize}
\end{frame}

% page 5
\begin{frame}{XALT Env. Vars for PKGS}
  \begin{itemize}
    \item Since the execution is happenning during XALT Tracking
    \item The following environment variables are defined
      \begin{itemize}
        \item XALT\_DIR: The root directory of where xalt\_record\_pkg
        \item XALT\_RUN\_UUID: The run uuid for the current R or
          Python program running
      \end{itemize}
    \item The R and Python Import hooks only collect data if
      XALT\_RUN\_UUID is defined
  \end{itemize}
\end{frame}

% page 6
\begin{frame}{\texttt{xalt\_record\_pkg} usage}
  \begin{itemize}
    \item The hook routine does the following
    \item Gets XALT\_DIR and XALT\_RUN\_UUID from env.
    \item Builds command path to \texttt{xalt\_record\_pkg} using
      XALT\_DIR
    \item The rest of the command line is:
      \begin{itemize}
        \item -u $<$run\_uuid$>$
        \item program $<$name$>$
        \item xalt\_run\_uuid $<$run\_uuid$>$
        \item package\_name $<$pkg\_name$>$
        \item package\_version $<$pkg\_version$>$
        \item package\_path $<$pkg\_path$>$
      \end{itemize}
  \end{itemize}
\end{frame}

% page 7
\begin{frame}{\texttt{xalt\_record\_pkg} execution}
  \begin{itemize}
    \item \texttt{xalt\_record\_pkg} builds a json string w/ data
    \item Every import will generate a record
    \item Why?
  \end{itemize}
\end{frame}

% page 8
\begin{frame}{\texttt{xalt\_record\_pkg} execution (II)}
  \begin{itemize}
    \item The hook code in R, Python is called dynamically
    \item There is nothing recording that can be called at the end.
    \item Originally XALT was going to make PKGS not be sampled,
    \item Also generate a start record
    \item This way import records would have something to connnect
      with.
    \item However this is a bad idea!
  \end{itemize}
\end{frame}

% page 9
\begin{frame}{\texttt{xalt\_record\_pkg} execution (III)}
  \begin{itemize}
    \item XALT needs an execution record stored to save import data
    \item But there are too much Python runs to store every one
    \item Instead XALT uses /dev/shm in a unique directory (UUID)
    \item This avoids overlap with other executions
    \item But why?
  \end{itemize}
\end{frame}

% page 10
\begin{frame}{Why write package import data to /dev/shm?}
  \begin{itemize}
    \item Speed
    \item Python and R can be sampled
    \item Data is only sent on the ``wire'' at end of program if sampled
    \item Delete data otherwise.
  \end{itemize}
\end{frame}

% page 11
\begin{frame}{Sent on the ``wire''?}
  \begin{itemize}
    \item Import records are saved on /dev/shm
    \item This is independent of \$XALT\_TRANSMISSION\_STYLE
    \item At the end of the execution the end record and import
      records are sent via \$XALT\_TRANSMISSION\_STYLE
    \item This is only if sampled.
  \end{itemize}
\end{frame}

% page 12
\begin{frame}{What to do with this data?}
  \begin{itemize}
    \item Find the list of heavily imported packages
    \item Find who is using conda python
    \item XALT won't know if something is imported but not used
    \item Track down heavily used packages and try to speed them up.
  \end{itemize}
\end{frame}

% page 13
\begin{frame}{Python hook: py\_src/xalt\_sitecustomize.py $\Rightarrow$ sitecustomize.py}
  \begin{itemize}
    \item Python 2 and 3 both look for  sitecustomize.py when starting
    \item Help from Riccardo Murri
    \item All Pythons uses sys.meta\_path to locate files to import
    \item Can register object to capture imports.
    \item Just add XALT's location of sitecustomize.py to PYTHONPATH
  \end{itemize}
\end{frame}

% page 14
\begin{frame}[fragile]
    \frametitle{Filtering python packages via site's Config.py}
 {\small
    \begin{semiverbatim}
python\_pkg\_patterns = [
\{ 'k\_s':'SKIP','kind':'path','patt': r"^[^/]" \},
\{ 'k\_s':'SKIP','kind':'name','patt': r"^_"    \},
\{ 'k\_s':'SKIP','kind':'name','patt': r".*\textbackslash." \},
\{ 'k\_s':'KEEP','kind':'path','patt': r".*/.local/" \},
]
    \end{semiverbatim}
}
\end{frame}

% page 20
\begin{frame}{Conclusions}
  \begin{itemize}
    \item We have a way to track imports from R and Python
    \item It works well but there are a few conflicts with
      sitecustomize.py
    \item We have yet more data to try to figure out what to do with.
  \end{itemize}
\end{frame}


% page 21
\begin{frame}{Future Topics?}
  \begin{itemize}
    \item Others?
  \end{itemize}
\end{frame}
%

%\input{./themes/license}

\end{document}
