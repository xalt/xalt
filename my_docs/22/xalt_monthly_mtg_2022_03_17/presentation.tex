\documentclass{beamer}

% You can also use a 16:9 aspect ratio:
%\documentclass[aspectratio=169]{beamer}
\usetheme{TACC16}
\usepackage{graphicx}

% It's possible to move the footer to the right:
%\usetheme[rightfooter]{TACC16}


% Detailed knowledge of application workload characteristics can
% optimize performance of current and future systems. This may sound
% daunting, with many HPC data centers hosting over 2,000 users running
% thousands of applications and millions of jobs per month.  XALT is an
% open source tool developed at the Texas Advanced Computing Center
% (TACC) that collects system usage information to quantitatively report
% how users are using your system. This session will explore the
% benefits of detailed application workload profiling and how the XALT
% tool has helped leading supercomputing sites unlock the power of their
% application usage data.

%% page 
%\begin{frame}{}
%  \begin{itemize}
%    \item
%  \end{itemize}
%\end{frame}
%
%% page 
%\begin{frame}[fragile]
%    \frametitle{}
% {\tiny
%    \begin{semiverbatim}
%    \end{semiverbatim}
%}
%  \begin{itemize}
%    \item
%  \end{itemize}
%
%\end{frame}



\begin{document}
\title[XALT]{Protecting XALT from Users}
\author{Robert McLay}
\date{March 17, 2022}

% page 1
\frame{\titlepage}

% page 2
\begin{frame}{XALT: Outline}
  \center{\includegraphics[width=.9\textwidth]{XALT_Sticker.png}}
  \begin{itemize}
    \item XALT is linking with every program that runs on the system
    \item Users make mistakes
    \item Need to protect XALT from user mistakes
    \item Show three protection examples 
  \end{itemize}
\end{frame}

% page 3
\begin{frame}{Three examples of protection}
  \begin{itemize}
    \item User's Expecting allocated memory to be zero'd
    \item User's mixing Fortran routine with C library routines badly
    \item XALT expecting well managed memory heap.
  \end{itemize}
\end{frame}


% page 4
\begin{frame}[fragile]
    \frametitle{How XALT works}
 {\tiny
    \begin{semiverbatim}
#include <stdio.h>
void myinit(int argc, char **argv)
\{ printf("This is run before main()\textbackslash{}n"); \}
void myfini()
\{ printf("This is run after main()\textbackslash{}n"); \}

__attribute__((section(".init_array"))) __typeof__(myinit) *__init = myinit;
__attribute__((section(".fini_array"))) __typeof__(myfini) *__fini = myfini;
    \end{semiverbatim}
}
  \begin{itemize}
    \item my\_docs/22/xalt\_monthly\_mtg\_2022\_03\_17/code/bad\_memory/ex1
  \end{itemize}
\end{frame}

% page 5
\begin{frame}[fragile]
    \frametitle{How XALT works (II)}
 {\small
    \begin{semiverbatim}
#include <stdio.h>
int main()
\{
  printf("Hello World!\textbackslash{}n");
  return 0;
\}

    \end{semiverbatim}
}
\end{frame}


% page 6
\begin{frame}[fragile]
    \frametitle{How XALT works (III)}
 {\small
    \begin{semiverbatim}
\$ ./try

Hello World!

\$ LD\_PRELOAD=./libxalt.so  ./try
This is run before main()
Hello World!
This is run after main()
    \end{semiverbatim}
}
  \begin{itemize}
    \item my\_docs/22/xalt\_monthly\_mtg\_2022\_03\_17/code/bad\_memory/ex1
  \end{itemize}

\end{frame}

% page 7
\begin{frame}{User's expecting memory to be zero'd when malloc() }
  \begin{itemize}
    \item Initially all memory is zero'd before program starts
    \item Note that pointer zero, integer zero and float zero are all zeros
    \item Link lists require a NULL pointer at end of list.
    \item Used memory is {\color{red} \emph{NOT}} zero'd for you in C.
    \item User programmed work w/o XALT, Failed with XALT.
  \end{itemize}
\end{frame}

% page 8
\begin{frame}[fragile]
    \frametitle{Example code clean/used memory}
 {\tiny
    \begin{semiverbatim}
\% cat try.c

#include <stdio.h>
#include <stdlib.h>
#define SZ 1000
int main()
\{
  int *a = (int *) malloc(SZ*sizeof(int));
  printf("Hello World! a:%d\textbackslash{}n",a[0]);
  return 0;
\}

\% cat xalt.c

#include <stdio.h>
#include <stdlib.h>
#include <string.h>
#define SZ 1000
void myinit(int argc, char **argv)
\{
  int i;
  int *a = (int*) malloc(SZ*sizeof(int));
  for (i = 0; i < SZ; ++i) a[i] = 15; 
  free(a);
  printf("This is run before main()\textbackslash{}n");
\}
__attribute__((section(".init_array"))) __typeof__(myinit) *__init = myinit;
    \end{semiverbatim}
}

  \begin{itemize}
    \item my\_docs/22/xalt\_monthly\_mtg\_2022\_03\_17/code/bad\_memory/ex2
  \end{itemize}
\end{frame}

% page 9
\begin{frame}[fragile]
    \frametitle{Example code clean/used memory(II)}
 {\small
   \begin{semiverbatim}
\% ./try

Hello World! {\color{blue} a:0}

\% LD_PRELOAD=./libxalt.so  ./try  ; echo
This is run before main()
Hello World! {\color{red} a:15}
This is run after main()
    \end{semiverbatim}
}
  \begin{itemize}
    \item my\_docs/22/xalt\_monthly\_mtg\_2022\_03\_17/code/bad\_memory/ex2
  \end{itemize}
\end{frame}

% page 10
\begin{frame}{XALT Fix: zero memory before free()}
  \begin{itemize}
    \item To protect XALT from broken user code
    \item XALT in myinit() zero's memory before free
    \item Note that non-MPI tracking does little allocation
    \item MPI tasks $>$ 127 init record $\Rightarrow$ much allocation
  \end{itemize}
\end{frame}



% page 11
\begin{frame}[fragile]
    \frametitle{XALT Fix: zero memory before free()}
 {\tiny
    \begin{semiverbatim}
\% cat try.c

#include <stdio.h>
#include <stdlib.h>
#define SZ 1000
int main()
\{
  int *a = (int *) malloc(SZ*sizeof(int));
  printf("Hello World! a:%d\textbackslash{}n",a[0]);
  return 0;
\}

\% cat xalt.c

#include <stdio.h>
#include <stdlib.h>
#include <string.h>
#define SZ 1000
void myinit(int argc, char **argv)
\{
  int i;
  int *a = (int*) malloc(SZ*sizeof(int));
  for (i = 0; i < SZ; ++i) a[i] = 15; 
  {\color{blue}{}memset((void *) a, 0, SZ*sizeof(int))};
  free(a);
  printf("This is run before main()\textbackslash{}n");
\}
__attribute__((section(".init_array"))) __typeof__(myinit) *__init = myinit;
    \end{semiverbatim}
}

  \begin{itemize}
    \item my\_docs/22/xalt\_monthly\_mtg\_2022\_03\_17/code/bad\_memory/ex3
  \end{itemize}
\end{frame}

% page 12
\begin{frame}[fragile]
    \frametitle{XALT Fix: zero memory before free() (II)}
 {\small
   \begin{semiverbatim}
\% ./try

Hello World! {\color{blue} a:0}

\% LD_PRELOAD=./libxalt.so  ./try  ; echo
This is run before main()
Hello World! {\color{blue} a:0}
This is run after main()
    \end{semiverbatim}
}
  \begin{itemize}
    \item my\_docs/22/xalt\_monthly\_mtg\_2022\_03\_17/code/bad\_memory/ex3
  \end{itemize}
\end{frame}


% page 13
\begin{frame}[fragile]
    \frametitle{Protecting XALT from Fortran mixed with C programs badly}
 {\tiny
    \begin{semiverbatim}
\% cat msg.f90
subroutine msg
   print *, "Hello World!"
end subroutine msg

\% nm try | grep msg
00000000000011c7 T msg_
    \end{semiverbatim}
}
  \begin{itemize}
    \item Normally fortran routines get a trailing underscore
      when compiled
    \item gfortran: -fno-underscoring
    \item ifort:    -assume nounderscore
    \item Will remove the trailing underscore
    \item Can make mixing C/Fortran easier
    \item Also make collisions with C library easier
  \end{itemize}
\end{frame}

% page 14
\begin{frame}{XALT uses libuuid}
  \begin{itemize}
    \item libuuid.so is used to get a unique identifier
    \item It uses libc's random()
    \item You only get one name per program
    \item Can't have two routines named random() 
    \item my\_docs/22/xalt\_monthly\_mtg\_2022\_03\_17/code/random/ex3


  \end{itemize}
\end{frame}

% page 15
\begin{frame}[fragile]
    \frametitle{Collision over random() routine}
 {\tiny
    \begin{semiverbatim}
\% cat try.f90

program tryMe
   implicit none 
   real*8 d, random
   print *, "Hello World!"
   d = {\color{blue}{}random(1.0, 2.0. 3.0)}
   print *, "d: ",d
end program tryMe

\% cat random.f90

real*8 function {\color{blue}{}random(a, b, c)}
   implicit none
   real*8 a, b, c
   print *, "In random(a, b, c)"
   random = a*b + c
end function random

\% cat xalt.c
#include <stdio.h>
#include <stdlib.h>
void myinit(int argc, char **argv)
\{
  long int a;
  printf("This is run before main()\textbackslash{}n");
  a = {\color{red}{}random()};
  printf("called random(): a: %ld\textbackslash{}n",a);
\}
__attribute__((section(".init_array"))) __typeof__(myinit) *__init = myinit;

\end{semiverbatim}
}
\end{frame}

% page 16
\begin{frame}[fragile]
    \frametitle{Collision over random() routine (II)}
 {\tiny
    \begin{semiverbatim}
\% ./try

 Hello World!
 In random(a, b, c)
 d:    5.0000000000000000     

\% LD_PRELOAD=./libxalt.so  ./try  ; echo
This is run before main()
 In random(a, b, c)
Segmentation fault
    \end{semiverbatim}
}
  \begin{itemize}
    \item The linker choses the user's fortran random() instead of the C lib
      random()
    \item The segfault happens because the fortan random() expects 3
      arguments
    \item xalt.c passes none.
  \end{itemize}

\end{frame}

% page 17
\begin{frame}{How to fix this issue}
  \begin{itemize}
    \item Other fortron program might do the same thing
    \item Trick: Use dlopen()/dlsym() to dynamicly link in libuuid.so
    \item At this point libuuid.so can ``see'' the fortran random()
      routine
    \item This trick solves many problems with libuuid
  \end{itemize}
\end{frame}












% page 20
\begin{frame}{Conclusions}
  \begin{itemize}
    \item We have a way to track imports from R and Python
    \item It works well but there are a few conflicts with
      sitecustomize.py
    \item We have yet more data to try to figure out what to do with.
  \end{itemize}
\end{frame}


% page 21
\begin{frame}{Future Topics?}
  \begin{itemize}
    \item Others?
  \end{itemize}
\end{frame}
%

%\input{./themes/license}

\end{document}
