\documentclass{beamer}

% You can also use a 16:9 aspect ratio:
%\documentclass[aspectratio=169]{beamer}
\usetheme{TACC16}
\usepackage{graphicx}

% It's possible to move the footer to the right:
%\usetheme[rightfooter]{TACC16}


% Detailed knowledge of application workload characteristics can
% optimize performance of current and future systems. This may sound
% daunting, with many HPC data centers hosting over 2,000 users running
% thousands of applications and millions of jobs per month.  XALT is an
% open source tool developed at the Texas Advanced Computing Center
% (TACC) that collects system usage information to quantitatively report
% how users are using your system. This session will explore the
% benefits of detailed application workload profiling and how the XALT
% tool has helped leading supercomputing sites unlock the power of their
% application usage data.

%% page 
%\begin{frame}{}
%  \begin{itemize}
%    \item
%  \end{itemize}
%\end{frame}
%
%% page 
%\begin{frame}[fragile]
%    \frametitle{}
% {\tiny
%    \begin{semiverbatim}
%    \end{semiverbatim}
%}
%  \begin{itemize}
%    \item
%  \end{itemize}
%
%\end{frame}



\begin{document}
\title[XALT]{Protecting XALT from Users}
\author{Robert McLay}
\date{April 21, 2022}

% page 1
\frame{\titlepage}

% page 2
\begin{frame}{XALT: Outline}
  \center{\includegraphics[width=.9\textwidth]{XALT_Sticker.png}}
  \begin{itemize}
    \item XALT is linking with every program that runs on the system
    \item Users will occasionally make mistakes
    \item Need to protect XALT from user mistakes
    \item Show more ways that XALT protects itself and users.
  \end{itemize}
\end{frame}

% page 3
\begin{frame}{Three examples of protection}
  \begin{itemize}
    \item User's bug hidden by zeroed memory initially
    \item User's mixing Fortran routine with C library routines badly
    \item XALT expecting well managed memory heap.
  \end{itemize}
\end{frame}


% page 4
\begin{frame}[fragile]
    \frametitle{How XALT works}
 {\tiny
    \begin{semiverbatim}
#include <stdio.h>
void myinit(int argc, char **argv)
\{ printf("This is run before main()\textbackslash{}n"); \}
void myfini()
\{ printf("This is run after main()\textbackslash{}n"); \}

__attribute__((section(".init_array"))) __typeof__(myinit) *__init = myinit;
__attribute__((section(".fini_array"))) __typeof__(myfini) *__fini = myfini;
    \end{semiverbatim}
}
  \begin{itemize}
    \item my\_docs/22/xalt\_monthly\_mtg\_2022\_03\_17/code/bad\_memory/ex1
  \end{itemize}
\end{frame}

% page 5
\begin{frame}[fragile]
    \frametitle{How XALT works (II)}
 {\small
    \begin{semiverbatim}
\% cat try.c

#include <stdio.h>
int main()
\{
  printf("Hello World!\textbackslash{}n");
  return 0;
\}

    \end{semiverbatim}
}
\end{frame}


% page 6
\begin{frame}[fragile]
    \frametitle{How XALT works (III)}
 {\small
    \begin{semiverbatim}
\$ ./try

Hello World!

\$ LD\_PRELOAD=./libxalt.so  ./try
This is run before main()
Hello World!
This is run after main()
    \end{semiverbatim}
}
  \begin{itemize}
    \item my\_docs/22/xalt\_monthly\_mtg\_2022\_03\_17/code/bad\_memory/ex1
  \end{itemize}

\end{frame}

% page 7
\begin{frame}{How XALT works (IV)}
  \begin{itemize}
    \item The xalt runtime library attaches to every program run on
      your system!
    \item I sometimes feel like I'm a developer on every program team.
    \item XALT programs are in the same namespace as the user's
      program (UGH!)
  \end{itemize}
\end{frame}



% page 8
\begin{frame}[fragile]
    \frametitle{Hiding XALT routine names from users}
 {\tiny
    \begin{semiverbatim}
\% nm \$LD\_PRELOAD| grep \_\_XALT\_build
0000000000009e80 T \_\_XALT\_buildEnvT\_xalt\_1\_5
000000000000a840 T \_\_XALT\_buildUserT\_xalt\_1\_5
00000000000163a0 T \_\_XALT\_buildXALTRecordT\_xalt\_1\_5
    \end{semiverbatim}
}
  \begin{itemize}
    \item XALT routine names are hidden by macros supplied in xalt\_obfuscate.h
  \end{itemize}

\end{frame}


% page 9
\begin{frame}{User's bug hidden by initially zeroed memory}
  \begin{itemize}
    \item Initially all memory is zeroed before program starts
    \item Note that pointer zero, integer zero and float zero are all
      zero bits
    \item Link lists require a NULL pointer at end of list.
    \item Used memory is {\color{red} \emph{NOT}} zeroed for you in C.
    \item User's program work w/o XALT, Failed with XALT.
  \end{itemize}
\end{frame}

% page 10
\begin{frame}[fragile]
    \frametitle{Example code clean/used memory}
 {\tiny
    \begin{semiverbatim}
\% cat try.c

#include <stdio.h>
#include <stdlib.h>
#define SZ 1000
int main()
\{
  int *a = (int *) malloc(SZ*sizeof(int));
  printf("Hello World! a:%d\textbackslash{}n",a[0]);
  return 0;
\}

\% cat xalt.c

#include <stdio.h>
#include <stdlib.h>
#include <string.h>
#define SZ 1000
void myinit(int argc, char **argv)
\{
  int i;
  int *a = (int*) malloc(SZ*sizeof(int));
  for (i = 0; i < SZ; ++i) a[i] = 15; 
  free(a);
  printf("This is run before main()\textbackslash{}n");
\}
\_\_attribute\_\_((section(".init\_array"))) \_\_typeof\_\_(myinit) *\_\_init = myinit;
    \end{semiverbatim}
}

  \begin{itemize}
    \item my\_docs/22/xalt\_monthly\_mtg\_2022\_03\_17/code/bad\_memory/ex2
  \end{itemize}
\end{frame}

% page 11
\begin{frame}[fragile]
    \frametitle{Example code clean/used memory(II)}
 {\small
   \begin{semiverbatim}
\% ./try

Hello World! {\color{blue} a:0}

\% LD\_PRELOAD=./libxalt.so  ./try  ; echo
This is run before main()
Hello World! {\color{red} a:15}
This is run after main()
    \end{semiverbatim}
}
  \begin{itemize}
    \item my\_docs/22/xalt\_monthly\_mtg\_2022\_03\_17/code/bad\_memory/ex2
  \end{itemize}
\end{frame}

% page 12
\begin{frame}{XALT Fix: zero memory before free()}
  \begin{itemize}
    \item To protect XALT from broken user code
    \item XALT in myinit() zero's memory before free
    \item Note that non-MPI tracking does little allocation
    \item MPI tasks $>$ 127 init record $\Rightarrow$ much allocation
  \end{itemize}
\end{frame}



% page 13
\begin{frame}[fragile]
    \frametitle{XALT Fix: zero memory before free()}
 {\tiny
    \begin{semiverbatim}
\% cat try.c

#include <stdio.h>
#include <stdlib.h>
#define SZ 1000
int main()
\{
  int *a = (int *) malloc(SZ*sizeof(int));
  printf("Hello World! a:%d\textbackslash{}n",a[0]);
  return 0;
\}

\% cat xalt.c

#include <stdio.h>
#include <stdlib.h>
#include <string.h>
#define SZ 1000
void myinit(int argc, char **argv)
\{
  int i;
  int *a = (int*) malloc(SZ*sizeof(int));
  for (i = 0; i < SZ; ++i) a[i] = 15; 
  {\color{blue}{}memset((void *) a, 0, SZ*sizeof(int))};
  free(a);
  printf("This is run before main()\textbackslash{}n");
\}
\_\_attribute\_\_((section(".init\_array"))) \_\_typeof\_\_(myinit) *\_\_init = myinit;
    \end{semiverbatim}
}

  \begin{itemize}
    \item my\_docs/22/xalt\_monthly\_mtg\_2022\_03\_17/code/bad\_memory/ex3
  \end{itemize}
\end{frame}

% page 14
\begin{frame}[fragile]
    \frametitle{XALT Fix: zero memory before free() (II)}
 {\small
   \begin{semiverbatim}
\% ./try

Hello World! {\color{blue} a:0}

\% LD_PRELOAD=./libxalt.so  ./try  ; echo
This is run before main()
Hello World! {\color{blue} a:0}
This is run after main()
    \end{semiverbatim}
}
  \begin{itemize}
    \item my\_docs/22/xalt\_monthly\_mtg\_2022\_03\_17/code/bad\_memory/ex3
  \end{itemize}
\end{frame}


% page 15
\begin{frame}[fragile]
    \frametitle{Protecting XALT from Fortran mixed with C programs badly}
 {\tiny
    \begin{semiverbatim}
\% cat msg.f90
subroutine msg
   print *, "Hello World!"
end subroutine msg

\% nm try | grep msg
00000000000011c7 T msg_
    \end{semiverbatim}
}
  \begin{itemize}
    \item Normally Fortran routines get a trailing underscore
      when compiled
    \item This can be disabled:
    \item Fortran: -fno-underscoring
    \item ifort:    -assume nounderscore
    \item Can make mixing C/Fortran easier
    \item Also make collisions with C library easier
  \end{itemize}
\end{frame}

% page 16
\begin{frame}{XALT uses libuuid}
  \begin{itemize}
    \item libuuid.so is used to get a unique identifier
    \item It uses libc's random()
    \item Can't have two routines named random() 
    \item my\_docs/22/xalt\_monthly\_mtg\_2022\_03\_17/code/random/ex3


  \end{itemize}
\end{frame}

% page 17
\begin{frame}[fragile]
    \frametitle{Collision over random() routine}
 {\tiny
    \begin{semiverbatim}
\% cat try.f90

program tryMe
   implicit none 
   real*8 d, random
   print *, "Hello World!"
   d = {\color{blue}{}random(1.0, 2.0. 3.0)}
   print *, "d: ",d
end program tryMe

\% cat random.f90

real*8 function {\color{blue}{}random(a, b, c)}
   implicit none
   real*8 a, b, c
   print *, "In random(a, b, c)"
   random = a*b + c
end function random

\% cat xalt.c
#include <stdio.h>
#include <stdlib.h>
void myinit(int argc, char **argv)
\{
  long int a;
  printf("This is run before main()\textbackslash{}n");
  a = {\color{red}{}random()};
  printf("called random(): a: %ld\textbackslash{}n",a);
\}
__attribute__((section(".init_array"))) __typeof__(myinit) *__init = myinit;

\end{semiverbatim}
}
\end{frame}

% page 18
\begin{frame}[fragile]
    \frametitle{Collision over random() routine (II)}
 {\tiny
    \begin{semiverbatim}
\% ./try

 Hello World!
 In random(a, b, c)
 d:    5.0000000000000000     

\% LD_PRELOAD=./libxalt.so  ./try  ; echo
This is run before main()
 In random(a, b, c)
Segmentation fault
    \end{semiverbatim}
}
  \begin{itemize}
    \item The linker chooses the user's Fortran random() instead of the C lib
      random()
    \item The segfault happens because the Fortran random() expects 3
      arguments
    \item the random() call in xalt.c passes none.
  \end{itemize}

\end{frame}

% page 19
\begin{frame}{How to fix this issue}
  \begin{itemize}
    \item Other Fortran program might do the same thing
    \item Trick: Use dlopen()/dlsym() to dynamically link in libuuid.so
    \item At this point libuuid.so can't ``see'' the Fortran random()
      routine
    \item This trick solves many problems with libuuid
  \end{itemize}
\end{frame}

% page 20
\begin{frame}{XALT is still susceptible to similar issues}
  \begin{itemize}
    \item XALT is now protected from a user's random() function
    \item But XALT is vulnerable some Fortran code replacing a c
      library routine
    \item We will just have to fix them as they come up
  \end{itemize}
\end{frame}


% page 21
\begin{frame}[fragile]
    \frametitle{Protecting XALT from badly managed memory heap}
 {\small
    \begin{semiverbatim}
void my\_free(void *ptr,int sz)
\{
  if (s\_start\_record \&\& ptr != NULL)
    \{
      memset(ptr, '\textbackslash{}0', sz);
      free(ptr);
    \}
\}
    \end{semiverbatim}
}
  \begin{itemize}
    \item Reporting an end record in myfini() requires memory
      allocations
    \item However some user programs can leave the heap broken
    \item XALT replaces free() with my\_free()
    \item Memory is only freed for a start record.
  \end{itemize}
\end{frame}

% page 22
\begin{frame}{Containers}
  \begin{itemize}
    \item Containers are chrooted environments
    \item They can have minimum setup and libraries
    \item Typically only what the user needs
    \item But not always what XALT needs.
    \item The xalt library needs many libraries libcrypto, libuuid,
      libcurl, etc.
  \end{itemize}
\end{frame}

% page 23
\begin{frame}{Containers (II)}
  \begin{itemize}
    \item For XALT to work at all in a container. LD\_PRELOAD must be
      set.
    \item And add XALT\_INSTALL\_DIR to container's available paths.
    \item So XALT during installation copies system libraries to the
      XALT\_INSTALL\_DIR
    \item XALT does a ldd on executables and libraries to find all the
      system libraries that need to be copied.
    \item This allows XALT to dlopen()/dlsym() libuuid.so to solve the
      problems mentioned before.
  \end{itemize}
\end{frame}


% page 24
\begin{frame}{MPI libraries disallow system() after main() returns}
  \begin{itemize}
    \item One of the version of IMPI, stopped supporting system()
      being called after main() finishes
    \item This was a disaster for collecting data from MPI programs.
    \item It even prevented collecting data from one task mpi programs
    \item This was a problem because XALT used C++ programs to collect data.
  \end{itemize}
\end{frame}

% page 25
\begin{frame}{Disallowed system() after main()}
  \begin{itemize}
    \item XALT needs Hash Table (AKA Dictionaries) to store many
      key-value pairs.
    \item C++ has built-in hash tables via the STL (unordered-maps)
    \item C does not.
    \item The XALT library (libxalt\_initialize.so) is written in C
    \item It would be a bad idea to have written the library in C++
  \end{itemize}
\end{frame}

% page 26
\begin{frame}{Solution}
  \begin{itemize}
    \item Re-write the XALT submission programs into C
    \item And include them in libxalt\_initialize.so
    \item Where to find a hash table implementation
    \item uthash
  \end{itemize}
\end{frame}

% page 27
\begin{frame}{UTHASH and friends}
  \begin{itemize}
    \item UTHASH: github: https://github.com/troydhanson/uthash
    \item UTHASH: Docs:
      https://troydhanson.github.io/uthash/userguide.html
    \item uthash is written in C via uthash.h (no library!)
  \end{itemize}
\end{frame}

% page 28
\begin{frame}[fragile]
    \frametitle{UTHASH}
 {\tiny
    \begin{semiverbatim}
// One C++ stmt:
userDT["start\_time"] = start\_time;

// Becomes several C stmts:
insert_key_double(&userDT, "start_time",  start_time);
// -->
void insert\_key\_double(S2D\_t** userDT, const char* name, double value)
{
  S2D\_t* entry = (S2D\_t *) XMALLOC(sizeof(S2D\_t));
  utstring\_new(entry->key);
  utstring\_bincpy(entry->key, name, strlen(name));

  entry->value = value;

  HASH\_ADD\_KEYPTR(hh, *userDT, utstring\_body(entry->key),
                    utstring\_len(entry->key), entry);
}
    \end{semiverbatim}
}
\end{frame}

% page 29
\begin{frame}{UTHASH rewrite}
  \begin{itemize}
    \item It took a couple weeks to rewrite the submission programs
      into routines
    \item The only system calls that still exist are the curl calls
      for transport.
  \end{itemize}
\end{frame}

% page 30
\begin{frame}{Support for filtering of packages}
  \begin{itemize}
    \item Python packages could be directly filtered through the
      sitecustomize.py file
    \item But not R or MATLAB packages
    \item All packages collection goes through the xalt\_record\_pkg
      program
  \end{itemize}
\end{frame}

% page 31
\begin{frame}[fragile]
    \frametitle{All package filtering}
  \begin{itemize}
    \item New pkg\_pattern array available in XALT 2.10.37+
    \item This goes in your site Config.py
    \item The pattern looks like this:
    \item Program:kind:pattern
  \end{itemize}
 {\tiny
    \begin{semiverbatim}
pkg_patterns = [
  ["SKIP",  r'^R:name:stats'],         # SKIP the R pkg named stats
  ["SKIP",  r'^R:name:base'],          # SKIP the R pkg named base
  ["SKIP",  r'^R:name:methods'],       # SKIP the R pkg named methods
  ["SKIP",  r'^python:name:\_.*'],      # SKIP all python name that start with an '\_'
  ["SKIP",  r'^python:path:[^/].*'],   # SKIP all python built-in packages
  ["SKIP",  r'^python:path:\textbackslash{}/home'],   # SKIP all python package in user locations
]
    \end{semiverbatim}
}

\end{frame}

% page 32
\begin{frame}{This package filtering could be added to ingestion}
  \begin{itemize}
    \item The same routine could be added to ingestion.
    \item If wanted it could be added in late May 2022.
  \end{itemize}
\end{frame}

% page 33
\begin{frame}{Conclusions}
  \begin{itemize}
    \item XALT has matured greatly from working with user programs
    \item Since the XALT library is in the same namespace as the user code 
    \item There is always a risk of routine collision.
  \end{itemize}
\end{frame}


% page 34
\begin{frame}{Future Topics?}
  \begin{itemize}
    \item Recent changes to importing json records
    \item Others?
  \end{itemize}
\end{frame}
%

%\input{./themes/license}

\end{document}
