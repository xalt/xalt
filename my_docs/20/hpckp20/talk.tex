\documentclass{beamer}

% You can also use a 16:9 aspect ratio:
%\documentclass[aspectratio=169]{beamer}
\usetheme{TACC16}
\usepackage{graphicx}

% It's possible to move the footer to the right:
%\usetheme[rightfooter]{TACC16}


% Detailed knowledge of application workload characteristics can
% optimize performance of current and future systems. This may sound
% daunting, with many HPC data centers hosting over 2,000 users running
% thousands of applications and millions of jobs per month.  XALT is an
% open source tool developed at the Texas Advanced Computing Center
% (TACC) that collects system usage information to quantitatively report
% how users are using your system. This session will explore the
% benefits of detailed application workload profiling and how the XALT
% tool has helped leading supercomputing sites unlock the power of their
% application usage data. 


\begin{document}
\title[XALT]{XALT: Job-Level Usage Data on Today's Supercomputers.}
\author{Robert McLay} 
\date{June 19, 2020} 

% page 1
\frame{\titlepage} 

% page 2
\begin{frame}{XALT: Outline}
  \center{\includegraphics[width=.9\textwidth]{XALT_Sticker.png}}
  \begin{itemize}
    \item What is XALT and what it is not?
    \item Brief History
    \item How it works: Three Parts
    \item What can you do with it?
    \item How can I help you?
  \end{itemize}
\end{frame}

% page 3
\begin{frame}{Understanding what your users are doing}
  \begin{itemize}
    \item What programs, libraries are your users using?
    \item What imports from R, MATLAB, Python?
    \item What are the top programs by core-hours? by counts? by users?
    \item System, User or Built by Other executables?
    \item Are Executables implemented in C/C++/Fortran?
    \item Track MPI task and/or Threading (\$OMP\_NUMTHREADS)
    \item Function Tracking
    \item Census Taker, Not a performance tool!
  \end{itemize}
\end{frame}

% page 4
\begin{frame}{Brief History}
  \begin{itemize}
    \item XALT was an U.S. NSF funded project (M. Fahey \& R. McLay)
    \item Work continued at TACC: too useful
    \item Originally only tracked MPI execution.
  \end{itemize}
\end{frame}

% page 5
\begin{frame}{Design Goals}
  \begin{itemize}
    \item Be extremely light-weight
    \item How many use a library or application?
    \item What functions are users calling in system modules
    \item Collect Data into a Database for analysis.
  \end{itemize}
\end{frame}

% page 6
\begin{frame}{How does XALT work?}
  \begin{itemize}
    \item LD Wrapper
    \item ELF Trick to track execution
    \item Signal handling
    \item Generate Json records
    \item Transport to DB
    \item Analyze database.
  \end{itemize}
\end{frame}

% page 7
\begin{frame}{Design: LD Wrapper}
  \begin{itemize}
    \item XALT wraps the linker to enable tracking of exec's
    \item The linker (ld) wrapper intercepts the user link line.
    \item Generate assembly code: key-value pairs
    \item Capture tracemap output from ld
    \item Transmit collected data in *.json format
    \item Optionally add codes that executes before main() and after
      main() completes for static builds.
  \end{itemize}
\end{frame}


% page 8
\begin{frame}{Signal handling}
  \begin{itemize}
    \item XALT registers signal handlers before main()
    \item SIGSEGV, SIGFPE, SIGTERM, etc
    \item User code can override with their own.
  \end{itemize}
\end{frame}

% page 9
\begin{frame}{Elf Trick (I)}
  \begin{itemize}
    \item ELF is the binary format for Linux
    \item ELF has many hooks
    \item XALT uses two hooks to run before/after main()
  \end{itemize}
\end{frame}

% page 10
\begin{frame}[fragile]
    \frametitle{ELF Trick (II)}
 {\small
    \begin{semiverbatim}
  #include <stdio.h>
  void myinit(int argc, char **argv)
  \{ fprintf(stderr, "This is run before main()\textbackslash{}n"); \}
  void myfini()
  \{ fprintf(stderr,"This is run after main()\textbackslash{}n");\}
  static \_\_attribute\_\_((section(".init_array")))
       typeof(myinit) *\_\_init = myinit;
  static \_\_attribute\_\_((section(".fini_array")))
       typeof(myfini) *\_\_fini = myfini;
    \end{semiverbatim}
}
\end{frame}

% page 11
\begin{frame}[fragile]
    \frametitle{ELF Trick (III)}
 {\small
    \begin{semiverbatim}
        \texttt{\% ./hello}
        Hello World!

        \texttt{\% LD\_PRELOAD=./libxalt.so ./hello}
        This is run before main()
        Hello World!
        This is run after main()
    \end{semiverbatim}
}
\end{frame}

% page 12
\begin{frame}{Transmission to DB}
  \begin{itemize}
    \item File: collect nightly/hourly/...
    \item Syslog: Use Syslog filtering (or ELK)
    \item Curl: send directly 
  \end{itemize}
\end{frame}

% page 13
\begin{frame}{Lmod to XALT connection (I)}
  \begin{itemize}
    \item Optional support to connect paths to modules
    \item Lmod spider walks entire module tree.
    \item Can build a reverse map from paths to modules
    \item Can map program \& libraries to modules.
    \item /opt/apps/i15/mv2\_2\_1/phdf5/1.8.14/lib/libhdf5.so.9
      $\Rightarrow$ phdf5/1.8.14(intel/15.02:mvapich2/2.1)
    \item Also helps with function tracking.
    \item Tmod Sites can still use Lmod to build the reverse map.
  \end{itemize}
\end{frame}


% page 14
\begin{frame}{Lmod to XALT connection (II)}
  \begin{itemize}
    \item Need XALT's ld before compiler's ld
    \item User loads a new compiler module?
    \item Lmod support path priority:
    \item \texttt{prepend\_path\{"PATH", "/.../xalt/bin", priority = 100\}}
  \end{itemize}
\end{frame}

% page 15
\begin{frame}[fragile]
    \frametitle{Lmod path priority (I)}
 {\small
    \begin{semiverbatim}
\$ type -a ld                     
ld is /opt/apps/xalt/xalt/bin/ld
ld is /opt/apps/gcc/8.3.0/bin/ld
ld is /bin/ld
    \end{semiverbatim}
}
\end{frame}

% page 16
\begin{frame}[fragile]
    \frametitle{Lmod path priority (II)}
 {\small
    \begin{semiverbatim}
\$ module load gcc/9.1.0; type -a ld                     
ld is /opt/apps/xalt/xalt/bin/ld
ld is /opt/apps/gcc/9.1.0/bin/ld
ld is /bin/ld
    \end{semiverbatim}
}
\end{frame}

% page 17
\begin{frame}{XALT LD Wrapper Support w/o Lmod}
  \begin{itemize}
    \item Move Compiler's ld to ld.x
    \item Small change in XALT's ld to find ld.x
    \item Must do this for every new compiler install.
    \item Or put xalt's bin path before compiler path in module
    \item Or treat every executable like \texttt{ls} or \texttt{ABAQUS}
  \end{itemize}
\end{frame}

% page 18
\begin{frame}{Installing XALT}
  \begin{itemize}
    \item Easy: \texttt{./configure ...; make install}
    \item Harder: Reverse Map from Lmod?
    \item Harder: Site config.py file
    \item Harder: Setup Transport Json records
    \item Harder: Setup VM to hold database
    \item Or: Setup your own way to handle the data
  \end{itemize}
\end{frame}

% page 19 
\begin{frame}{Site config.py (I)}
  \begin{itemize}
    \item Each site must configure to match their setup
    \item Compute node names?
    \item What executables to track or ignore?
    \item What python packages to track or ignore?
    \item What sampling rules to use?
  \end{itemize}
\end{frame}

% page 20
\begin{frame}{Site config.py (II)}
  \begin{itemize}
    \item XALT use config.py to create *.h, *.lex *.py files during build.
    \item Provides xalt\_configuration\_report C++ program to know how configured.
    \item Config.py file only used when building XALT.
    \item Any changes to Config.py require a re-install of XALT.
  \end{itemize}
\end{frame}

% page 21
\begin{frame}{Hostname, Path and Env Filtering}
  \begin{itemize}
    \item Uses FLEX to compile in patterns
    \item Use regex expression to control what to keep and ignore.
    \item These Flex $\Rightarrow$ C routine provide fast regex parsing.
  \end{itemize}
\end{frame}

% page 22
\begin{frame}[fragile]
    \frametitle{TACC\_config.py}
 {\small
    \begin{semiverbatim}
hostname\_patterns = [
   ['KEEP', '^c[0-9][0-9][0-9]-[0-9][0-9][0-9]\textbackslash{}..*']
  ['KEEP', '^nid[0-9][0-9][0-9][0-9][0-9].*']
]
path\_patterns = [
    ['PKGS',  r'.*\textbackslash{}/python[0-9.]*'],
    ['PKGS',  r'.*\textbackslash{}/R'],
    ['KEEP',  r'^/\textbackslash{}usr\textbackslash{}/bin\textbackslash{}/ddt'],
    ['SKIP',  r'^\textbackslash{}/bin\textbackslash{}/.*'],
]
env\_patterns = [
    [ 'SKIP', r'^MKLROOT=.*' ],
    [ 'SKIP', r'^MKL\_DIR=.*' ],
    [ 'KEEP', r'^I\_MPI\_INFO\_NUMA\_NODE\_NUM=.*'],
]
    \end{semiverbatim}
}
\end{frame}

% page 23
\begin{frame}{How sampling works}
  \begin{itemize}
    \item Changed design to deal with the overload of XALT data
    \item Only generate records if plan to save.
    \item All Non-mpi executions only produce end records.
    \item Small MPI execution sample, Large MPI executions record.
  \end{itemize}
\end{frame}

% page 24
\begin{frame}{Sampling Non-MPI programs}
  \begin{itemize}
    \item XALT has sampling rules (site configurable!)
    \item TACC rules are:
    \item 0 mins $<$ 30 mins $\Rightarrow$ 0.01\% recorded  
    \item 30 mins $<$ 120 mins $\Rightarrow$ 1\% recorded  
    \item 120 mins $<  \infty \Rightarrow$ 100\% recorded
    \item Can now track/sample perl, awk, sed, gzip etc
  \end{itemize}
\end{frame}

% page 25
\begin{frame}{Sampling MPI programs}
  \begin{itemize}
    \item Some users are using many short MPI programs to train Deep
      Learning engine
    \item TACC rules are:
    \item Task counts $<$ 128 tasks are sampled
      \begin{itemize}
        \item 0 mins $<$ 15 mins $\Rightarrow$ 0.01\% recorded  
        \item 15 mins $<$ 30 mins $\Rightarrow$ 1\% recorded  
        \item 30 mins $<  \infty \Rightarrow$ 100\% recorded
      \end{itemize}
    \item Task counts $>=$ 128 task are always recorded independent of runtime.
    \item Need to Capture long running MPI progs that never end.
  \end{itemize}
\end{frame}

% page 26
\begin{frame}{Using XALT Data}
  \begin{itemize}
    \item Targeted Outreach: Who will be affected
    \item Largemem Queue Overuse
    \item XALT and TACC-Stats
    \item Who is running NWChem or ...?
    \item Function Tracking: Who or What is using MPI-3?      
  \end{itemize}
\end{frame}

% page 27
\begin{frame}{Who is using MPI-3: MPI\_I*}
    What codes link in MPI-3 routines.  (Not necessarily Run)
    \begin{tabular}{|l|r|r|}
        \hline
        Function Name            & N Users    & N Progs \\\hline\hline
        MPI\_Ibarrier            &  8         & 4       \\\hline
        MPI\_Ialltoall           & 24         & 4       \\\hline
        MPI\_Ineighbor\_alltoall &  4         & 3       \\\hline
    \end{tabular}

\end{frame}

% page 28
\begin{frame}{What is new with XALT?}
  \begin{itemize}
    \item Tracking R, Python, MATLAB
    \item Signal handler
    \item Optionally Track GPU Usage
    \item Track Singularity Container Usage
    \item Removed three system calls for improved speed
    \item \texttt{xalt\_configuration\_report}
  \end{itemize}
\end{frame}

% page 29
\begin{frame}{Tracking R packages}
  \begin{itemize}
    \item XALT can now track R package usage
    \item James McComb \& Michael Scott from IU developed the R part
    \item They do this by intercepting the ``imports''
  \end{itemize}
\end{frame}

% page 30
\begin{frame}{Tracking Python packages}
  \begin{itemize}
    \item Help from Riccardo Murri
    \item sitecustomize.py
    \item It is run by any Python if found.
    \item All Pythons uses sys.meta\_path to locate files to import
    \item Can register object to capture imports.
    \item Just add location to PYTHONPATH
  \end{itemize}
\end{frame}

% page 31
\begin{frame}[fragile]
    \frametitle{Filtering python packages}
 {\small
    \begin{semiverbatim}
\{ 'k\_s':'SKIP','kind':'path','patt': r"^[^/]" \},
\{ 'k\_s':'SKIP','kind':'name','patt': r"^_"    \},
\{ 'k\_s':'SKIP','kind':'name','patt': r".*\textbackslash." \},
\{ 'k\_s':'KEEP','kind':'path','patt': r".*/.local/" \},
    \end{semiverbatim}
}
\end{frame}

% page 32
\begin{frame}{New program: xalt\_extract\_record}
  \begin{itemize}
    \item This program reads the watermark.
    \item Find out who built this program on what machine
    \item Find out what modules where used.
    \item Where was it build. 
  \end{itemize}
\end{frame}

% page 33
\begin{frame}[fragile]
    \frametitle{Example of xalt\_extract\_record output}
 {\small
    \begin{semiverbatim}
****************************************
XALT Watermark: hello
****************************************
Build\_CWD            /home/user/t/hello
Build\_Epoch          1510257139.4624
Build\_LMFILES        /apps/mfiles/intel/17.0.4.lua:...
Build\_LOADEDMODULES  intel/18.0.4:impi/18.0.3:TACC:...
Build\_OS             Linux 3.10.0-514.26.2.el7.x86_64
Build\_Syshost        stampede2
Build\_UUID           586d5943-67eb-480b-a2fe-35e87a1f22c7
Build\_User           mclay
Build\_compiler       icc
Build\_date           Fri Jun 09 13:52:19 2019
Build\_host           c455-011.stampede2.tacc.utexas.edu
XALT\_Version         2.7
    \end{semiverbatim}
}
\end{frame}

% page 34
\begin{frame}{New Feature: Track GPU usage}
  \begin{itemize}
    \item Optionally, XALT can know if a GPU was used.
    \item XALT will only know if one or more GPU's were accessed
    \item No performance data
    \item Thanks to Scott McMillan from NVIDIA for the contribution.
  \end{itemize}
\end{frame}

% page 35
\begin{frame}{New Feature: Track Singularity Container Usage}
  \begin{itemize}
    \item Sites can configure their Singularity script to include XALT
    \item It works well with syslog, file  or curl transfer of data
    \item Thanks to Scott McMillan from NVIDIA for the contribution.
  \end{itemize}
\end{frame}

% page 36
\begin{frame}[fragile]
    \frametitle{Debugging XALT}
 {\tiny
    \begin{semiverbatim}
\$ XALT\_TRACING=yes XALT\_SAMPLING=no ./hello

myinit(0/1,LD_PRELOAD,/path/to/hello)\{
    -> Setting up signals
    -> Leaving myinit
\}

Hello World!

myfini(0/1,LD_PRELOAD,/path/to/hello)\{
    -> XALT_SAMPLING = "no" All programs tracked!
  Recording State at end of scalar user program:
    XALT_EXECUTABLE_TRACKING=no /path/to/xalt_run_submission ...

  xalt_run_submission(.zzz) \{
    building UUID: ...
    Extracted recordT from executable
    Built userT, userDT, scheduler: SLURM
    Using XALT_TRANSMISSION_STYLE: file
    cmdlineA: ["./hello"]
    Built json string
  \}
    -> leaving myfini
\}
    \end{semiverbatim}
}
\end{frame}

% page 37
\begin{frame}[fragile]
    \frametitle{XALT Doc usage by Country}
    \center{\includegraphics[width=.9\textwidth]{XALT_usage_by_country}}
\end{frame}

% page 38
\begin{frame}[fragile]
    \frametitle{XALT Doc usage by City}
    \center{\includegraphics[width=.9\textwidth]{XALT_usage_by_city}}
\end{frame}

% page 39
\begin{frame}{Conclusion}
  \center{\includegraphics[width=.9\textwidth]{XALT_Sticker.png}}
  \begin{itemize}
    \item Lmod:
      \begin{itemize}
        \item Source: github.com/TACC/lmod.git, lmod.sf.net
        \item Documentation: lmod.readthedocs.org
      \end{itemize}
    \item XALT:
      \begin{itemize}
        \item Source: github.com/xalt/xalt.git, xalt.sf.net
        \item Documentation: XALT 2 $\Rightarrow$ xalt.readthedocs.org
        \item Join mailing list: https://sourceforge.net/projects/xalt/lists/xalt-users
      \end{itemize}
  \end{itemize}
\end{frame}

%\input{./themes/license}

\end{document}
