\documentclass{beamer}

% You can also use a 16:9 aspect ratio:
%\documentclass[aspectratio=169]{beamer}
\usetheme{TACC16}

% It's possible to move the footer to the right:
%\usetheme[rightfooter]{TACC16}


% Detailed knowledge of application workload characteristics can
% optimize performance of current and future systems. This may sound
% daunting, with many HPC data centers hosting over 2,000 users running
% thousands of applications and millions of jobs per month.  XALT is an
% open source tool developed at the Texas Advanced Computing Center
% (TACC) that collects system usage information to quantitatively report
% how users are using your system. This session will explore the
% benefits of detailed application workload profiling and how the XALT
% tool has helped leading supercomputing sites unlock the power of their
% application usage data. 


\begin{document}
\title[XALT]{XALT: Understanding HPC Usage via Job Level Collection}
\author{Robert McLay} 
\date{June 19, 2019} 

% page 1
\frame{\titlepage} 

% page 2
\begin{frame}{XALT: Outline}
  \begin{itemize}
    \item What is XALT and what it is not?
    \item How it works
    \item What is new with XALT?
    \item What can you do with it?
    \item How can I help you?
  \end{itemize}
\end{frame}

% page 3
\begin{frame}{XALT: What runs on the system}
  \begin{itemize}
    \item Was a NSF Funded project.
    \item A Census of what programs and libraries are run
    \item Running at TACC, NICS, U. Florida, KAUST, LLNL, ...
    \item Integrates with TACC-Stats.
    \item Has commercial support from Ellexus 
    \item Not tracking performance only existance 
  \end{itemize}
\end{frame}

% page 4
\begin{frame}{Understanding what your users are doing}
  \begin{itemize}
    \item What programs, libraries are your users using?
    \item What are the top programs by core-hours? by counts ? by users?
    \item Are they building their own programs or using someone elses?
    \item Are Executables implemented in C/C++/Fortran?
    \item Track MPI: tasks? nodes?
    \item Track Threading via \$OMP\_NUMTHREADS
  \end{itemize}
\end{frame}

% page 5
\begin{frame}{History of XALT}
  \begin{itemize}
    \item Mark Fahey (was NICS, now ANL): ALT-D (MPI only)
    \item Robert McLay (TACC) Lariat (MPI only)
    \item Reuben Budiardja (was NICS now ORL)
    \item ALT-D $+$ Lariat $\Rightarrow$ XALT 1: (MPI only)
    \item XALT 2: All programs
  \end{itemize}
\end{frame}

% page 6
\begin{frame}{Design Goals}
  \begin{itemize}
    \item Be extremely light-weight
    \item How many use a library or application?
    \item Collect Data into a Database for analysis.
  \end{itemize}
\end{frame}

% page 7
\begin{frame}{Design: Linker}
  \begin{itemize}
    \item XALT wraps the linker to enable tracking of exec's
    \item The linker (ld) wrapper intercepts the user link line.
    \item Generate assembly code: key-value pairs
    \item Capture tracemap output from ld
    \item Transmit collected data in *.json format
    \item Optionally add codes that executes before main() and after main() completes
  \end{itemize}
\end{frame}

% page 8
\begin{frame}{Design: Transmission to DB}
  \begin{itemize}
    \item File: collect nightly
    \item Syslog: Use Syslog filtering (or ELK)
  \end{itemize}
\end{frame}

% page 9
\begin{frame}{Lmod to XALT connection}
  \begin{itemize}
    \item Lmod spider walks entire module tree.
    \item Can build a reverse map from paths to modules
    \item Can map program \& libraries to modules.
    \item /opt/apps/i15/mv2\_2\_1/phdf5/1.8.14/lib/libhdf5.so.9
      $\Rightarrow$ phdf5/1.8.14(intel/15.02:mvapich2/2.1)
    \item Also helps with function tracking.
    \item Tmod Sites can still use Lmod to build the reverse map.
  \end{itemize}
\end{frame}

% page 10
\begin{frame}{Using XALT Data}
  \begin{itemize}
    \item Targetted Outreach: Who will be affected
    \item Largemem Queue Overuse
    \item XALT and TACC-Stats
    \item Who is running NWChem or ...? 
  \End{itemize}
\end{frame}

% page 11
\begin{frame}{Tracking Non-mpi jobs (I)}
  \begin{itemize}
    \item Originally we tracked only MPI Jobs
    \item By hijacking mpirun etc.
    \item Now we can use ELF binary format to track jobs
  \end{itemize}
\end{frame}

% page 12
\begin{frame}[fragile]
    \frametitle{ELF Binary Format Trick}
 {\small
    \begin{semiverbatim}
void myinit(int argc, char **argv)
\{
  /* ... */
\}
void myfini()
\{
  /* ... */
\}
  static __attribute__((section(".init_array")))
       typeof(myinit) *__init = myinit;
  static __attribute__((section(".fini_array")))
       typeof(myfini) *__fini = myfini;
    \end{semiverbatim}
}
\end{frame}

% page 13
\begin{frame}{Using the ELF Binary Format Trick}
  \begin{itemize}
    \item This C code can be compiled and linked in through the hijacked linker
    \item It can also be used with \texttt{LD\_PRELOAD}
    \item By default we only use \texttt{LD\_PRELOAD} but can do both.
  \end{itemize}
\end{frame}

% page 14
\begin{frame}{Challenges (I)}
  \begin{itemize}
    \item Do not want to track every mv, cp, etc
    \item Only want to track some executables on compute nodes
    \item Do not want to get overwhelmed by the data. 
  \end{itemize}
\end{frame}

% page 15
\begin{frame}{Answers}
  \begin{itemize}
    \item XALT Tracking only when told to
    \item Compute node only by host name filtering
    \item Executable Filter based on Path
    \item Protection against closing stderr before fini.
    \item Sampling of serial programs.
    \item Site configurable!
  \end{itemize}
\end{frame}

% page 16
\begin{frame}{Path Filtering}
  \begin{itemize}
    \item Uses FLEX to compile in patterns
    \item Use regex expression to control what to keep and ignore.
    \item Three files containing regex patterns, converted to code.
    \item Accept List Tests: Track /usr/bin/ddt, /bin/tar, /usr/bin/perl
    \item Ignore List Tests: /usr/bin, /bin, /sbin, ...
  \end{itemize}
\end{frame}

% page 17
\begin{frame}[fragile]
    \frametitle{TACC\_config.py}
 {\small
    \begin{semiverbatim}
hostname\_patterns = [
  ['KEEP', '^c[0-9][0-9][0-9]-[0-9][0-9][0-9]\..*']
  ]
path\_patterns = [
    ['PKGS',  r'.*/python[0-9][^/][^/]*'],
    ['PKGS',  r'.*/R'],
    ['KEEP',  r'^/usr/bin/ddt'],
    ['SKIP',  r'^/usr/.*'],
    ['SKIP',  r'^/bin/.*'],
  ]
env\_patterns = [
    [ 'SKIP', r'^MKLROOT=.*' ],
    [ 'SKIP', r'^MKL\_DIR=.*' ],
    [ 'KEEP', r'^I\_MPI\_INFO\_NUMA\_NODE\_NUM=.*'],
  ]
    \end{semiverbatim}
}
\end{frame}

% page 18
\begin{frame}{Speeding up XALT 2}
  \begin{itemize}
    \item Minimal impact on jobs ($>$ 0.09 secs)
    \item All Non-MPI programs only produce end record
    \item This supports sampling!
  \end{itemize}
\end{frame}

% page 19
\begin{frame}{Sampling Non-MPI programs}
  \begin{itemize}
    \item XALT has sampling rules (site configurable!)
    \item 0 mins $<$ 5 mins $\Rightarrow$ 0.01\% recorded  
    \item 5 mins $<$ 10 mins $\Rightarrow$ 1\% recorded  
    \item 10 mins $<  \infty \Rightarrow$ 100\% recorded
    \item Can now track perl, awk, sed, gzip etc
  \end{itemize}
\end{frame}

% page 20
\begin{frame}{What is new with XALT?}
  \begin{itemize}
    \item Tracking R, Python, MATLAB
    \item Signal handler
    \item Optionally Track GPU Usage
    \item Track Singularity Container Usage
    \item Removed two system calls
  \end{itemize}
\end{frame}

% page 21
\begin{frame}{Tracking R packages}
  \begin{itemize}
    \item XALT 2 can now track R package usage
    \item James McComb \& Michael Scott from IU developed the R part
    \item They do this by intercepting the ``imports''
    \item Plan to support Python and MATLAB later.
  \end{itemize}
\end{frame}

% page 22
\begin{frame}{New program: xalt\_extract\_record}
  \begin{itemize}
    \item This program reads the watermark.
    \item Find out who built this program on what machine
    \item Find out what modules where used.
  \end{itemize}
\end{frame}

% page 23
\begin{frame}[fragile]
    \frametitle{Example of xalt\_extract\_record output}
 {\small
    \begin{semiverbatim}
****************************************
XALT Watermark: hello
****************************************
Build\_CWD                /home/user/t/hello
Build\_Epoch              1510257139.4624
Build\_LMFILES            /opt/apps/modulefiles/intel/17.0.4.lua:...
Build\_LOADEDMODULES      intel/18.0.4:impi/18.0.3:python/2.7.13:TACC:...
Build\_OS                 Linux 3.10.0-514.26.2.el7.x86_64
Build\_Syshost            stampede2
Build\_UUID               586d5943-67eb-480b-a2fe-35e87a1f22c7
Build\_User               mclay
Build\_compiler           icc
Build\_date               Fri Jun 09 13:52:19 2019
Build\_host               c455-011.stampede2.tacc.utexas.edu
XALT\_Version             2.7
    \end{semiverbatim}
}
\end{frame}


% page 24
\begin{frame}{XALT Signal Handler}
  \begin{itemize}
    \item Program that fail with SEGV, ILL, FPE ... produce an XALT record.
    \item But not SIGINT.
    \item The signal handlers are assigned before main() $\Rightarrow$
      Doesn't interfere with user handlers
  \end{itemize}
\end{frame}

% page 25
\begin{frame}{New Feature: Track GPU usage}
  \begin{itemize}
    \item Optionally, XALT can know if a GPU was used.
    \item XALT will only know if one or more GPU's were accessed
    \item No performance data
    \item Thanks to Scott McMillan from NVIDIA for the contribution.
  \end{itemize}
\end{frame}

% page 26
\begin{frame}{New Feature: Track Singularity Container Usage}
  \begin{itemize}
    \item Sites can configure their Singularity script to include XALT
    \item It works well with syslog or file transfer of data
    \item Thanks to Scott McMillan from NVIDIA for the contribution.
  \end{itemize}
\end{frame}

% page 27
\begin{frame}{Removed two system calls during Tracking}
  \begin{itemize}
    \item Reads /proc/$PID/maps instead of running ldd.
    \item Uses the vendor note to hold the XALT watermark.
    \item Improves XALT penalty from 0.01 to 0.001 seconds.
  \end{itemize}
\end{frame}

% page 28
\begin{frame}{Conclusion}
  \begin{itemize}
    \item Lmod:
      \begin{itemize}
        \item Source: github.com/TACC/lmod.git, lmod.sf.net
        \item Documentation: lmod.readthedocs.org
      \end{itemize}
    \item XALT:
      \begin{itemize}
        \item Source: github.com/xalt/xalt.git, xalt.sf.net
        \item Documentation: XALT 2 $\Rightarrow$ xalt.readthedocs.org
      \end{itemize}
  \end{itemize}
\end{frame}

%\input{./themes/license}

\end{document}
