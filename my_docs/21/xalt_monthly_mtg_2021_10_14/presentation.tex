\documentclass{beamer}

% You can also use a 16:9 aspect ratio:
%\documentclass[aspectratio=169]{beamer}
\usetheme{TACC16}
\usepackage{graphicx}

% It's possible to move the footer to the right:
%\usetheme[rightfooter]{TACC16}


% Detailed knowledge of application workload characteristics can
% optimize performance of current and future systems. This may sound
% daunting, with many HPC data centers hosting over 2,000 users running
% thousands of applications and millions of jobs per month.  XALT is an
% open source tool developed at the Texas Advanced Computing Center
% (TACC) that collects system usage information to quantitatively report
% how users are using your system. This session will explore the
% benefits of detailed application workload profiling and how the XALT
% tool has helped leading supercomputing sites unlock the power of their
% application usage data. 


\begin{document}
\title[XALT]{XALT Monthly Meeting: Building a Reverse Map Paths $\Rightarrow$ Module Names}
\author{Robert McLay} 
\date{Oct. 14, 2021} 

% page 1
\frame{\titlepage} 

% page 2
\begin{frame}{XALT: Outline}
  \center{\includegraphics[width=.9\textwidth]{XALT_Sticker.png}}
  \begin{itemize}
    \item Setting up modulefiles to build a Reverse Map
    \item The Reverse Map connects full paths to module names
    \item It is not required but useful.
    \item This works even if your site uses Tmod.
    \item Just use Lmod to build the reverse Map.
  \end{itemize}
\end{frame}

% page 3
\begin{frame}{Goal: Map full paths to module names}
  \begin{itemize}
    \item \texttt{\$LMOD\_DIR/spider -o xalt\_rmapT \$MODULEPATH $>$ xalt\_rmapT.json}
    \item TACC builds this file on cluster and transfers it to VM.
  \end{itemize}
\end{frame}

% page 4
\begin{frame}[fragile]
    \frametitle{xalt\_rmapT.json}
 {\tiny
    \begin{semiverbatim}
\{
 "reverseMapT": \{
     "/apps/cmake/3.16.1": "cmake/3.16.1",
     "/apps/cmake/3.16.1/bin": "cmake/3.16.1",
     "/apps/gcc/9.1.0": "gcc/9.1.0",
     "/apps/gcc/9.1.0/bin": "gcc/9.1.0",
     "/apps/gcc/9.1.0/lib": "gcc/9.1.0",
     "/apps/gcc/9.1.0/lib64": "gcc/9.1.0",
     "/apps/intel18/impi18_0/phdf5/1.10": "phdf5/1.10.4(intel/18.0:impi/18.0.0)",
     "/apps/intel18/impi18_0/phdf5/1.10/bin": "phdf5/1.10.4(intel/18.0:impi/18.0.0)",
     "/apps/intel18/impi18_0/phdf5/1.10/lib": "phdf5/1.10.4(intel/18.0:impi/18.0.0)",
     ...
   \}
 \}
    \end{semiverbatim}
}
\end{frame}

% page 5
\begin{frame}{Lmod configuration}
  \begin{itemize}
    \item Lmod uses LMOD\_SITE\_NAME to know what the name of your cluster
    \item For example we use TACC as our site name.  
    \item But lets assume that your site is BLUE.
  \end{itemize}
\end{frame}

% page 6
\begin{frame}{Full paths are remembered from 3 Env. Vars.}
  \begin{itemize}
    \item Changes to PATH
    \item Changes to LD\_LIBRARY\_PATH
    \item Setting $<$SITE$>$\_$<$PKGName$>$\_DIR
    \item This last one is very important.
  \end{itemize}
\end{frame}

% page 7
\begin{frame}{$<$SITE$>$\_$<$PKGName$>$\_DIR}
  \begin{itemize}
    \item Let's assume that your site name is BLUE.
    \item Each modulefile should have:
      \texttt{setenv("BLUE\_$<$PKGName$>$\_DIR", "/path/to/package"}
    \item PHDF5:     \texttt{setenv("BLUE\_HDF5\_DIR", "/apps/intel19/impi19/phdf5/1.10")}

  \end{itemize}
\end{frame}

% page 8
\begin{frame}{Converting a PATH to a Module Name}
  \begin{itemize}
    \item Full paths are supplied by XALT records
    \item XALT uses xalt\_rmapT.json file to map paths to names
    \item It uses the three paths described before.
    \item Given a full path \texttt{/apps/intel19/impi19/phdf5/1.10/bin/h5dump}
    \item XALT removes one directory at a time to get an exact match
    \item Thus mapping path to module phdf5/1.10
    \item If no match is found then \texttt{module\_name} is NULL
    \item Mapping is done at ingestion by xalt\_file\_to\_db.py or xalt\_syslog\_to\_db.py
  \end{itemize}
\end{frame}


% page 9
\begin{frame}{Conclusions}
  \begin{itemize}
    \item Thanks for listening
    \item Questions?
  \end{itemize}
\end{frame}

%\input{./themes/license}

\end{document}
