\documentclass{beamer}

% You can also use a 16:9 aspect ratio:
%\documentclass[aspectratio=169]{beamer}
\usetheme{TACC16}
\usepackage{graphicx}

% It's possible to move the footer to the right:
%\usetheme[rightfooter]{TACC16}


% Detailed knowledge of application workload characteristics can
% optimize performance of current and future systems. This may sound
% daunting, with many HPC data centers hosting over 2,000 users running
% thousands of applications and millions of jobs per month.  XALT is an
% open source tool developed at the Texas Advanced Computing Center
% (TACC) that collects system usage information to quantitatively report
% how users are using your system. This session will explore the
% benefits of detailed application workload profiling and how the XALT
% tool has helped leading supercomputing sites unlock the power of their
% application usage data. 


\begin{document}
\title[XALT]{XALT Monthly Meeting: Using \$XALT\_TRACING to track XALT Behavior }
\author{Robert McLay} 
\date{Nov. 11, 2021} 

% page 1
\frame{\titlepage} 

% page 2
\begin{frame}{XALT: Outline}
  \center{\includegraphics[width=.9\textwidth]{XALT_Sticker.png}}
  \begin{itemize}
    \item XALT provides a way to debug its behavior
    \item XALT\_TRACING=link  icc -o try try.c
    \item XALT\_TRACING=run ./user\_program ...
    \item It shows what decision XALT is making.
  \end{itemize}
\end{frame}

% page 3
\begin{frame}[fragile]
    \frametitle{Linking w/ XALT}
 {\tiny
    \begin{semiverbatim}
\$ XALT_TRACING=link gcc -o hello hello.c

14:57:32 XALT Tracing Activated for version: xalt-2.10.31

14:57:32 Starting ld

14:57:32 find_real_command: Searching for the real: ld

14:57:32 find_real_command: found /opt/apps/gcc/8.3.0/bin/ld

14:57:32 COMP_T: {"compiler":"gcc","compilerPath":"/opt/apps/gcc/8.3.0/bin/gcc","link_line":["gcc","-o","hello","hello.c"],"parentProcs":["zsh:/usr/bin/zsh","sshd:","sshd:","sshd:"]}

14:57:32 SYSHOST: frontera

14:57:32 WRKDIR: /tmp/user_dd798a17-2dda-411f-8029-077bf67ad3a4_FChre5

14:57:32 EPOCH 1636142252.7051

14:57:32 Using ld to find functions:  /opt/apps/gcc/8.3.0/bin/ld ...

14:57:32 Not adding XALT initialize routines to user code

14:57:32 /opt/apps/gcc/8.3.0/bin/ld ...

14:57:32 XALT_TRANSMISSION_STYLE file
    Using USE_HOME method to store json file
    Wrote json link file : 
   \end{semiverbatim}
}
\end{frame}

% page 4
\begin{frame}[fragile]
    \frametitle{Tracing on login node}
 {\tiny
    \begin{semiverbatim}
staff.fta(3013)% XALT\_TRACING=run ./hello                                                    
---------------------------------------------
 Date:          Fri Nov  5 13:16:26 2021
 XALT Version:  xalt-2.10.31
 Nodename:      staff.frontera.tacc.utexas.edu
 System:        Linux
 Release:       3.10.0-1160.45.1.el7.x86_64
 O.S. Version:  #1 SMP Wed Oct 13 17:20:51 UTC 2021
 Machine:       x86_64
---------------------------------------------

myinit(0/1,LD\_PRELOAD,/home/user/t/xalt/hello)\{
  Test for \_\_XALT\_INITIAL\_STATE\_\_: "(NULL)", STATE: "LD_PRELOAD"
  Test for XALT\_EXECUTABLE\_TRACKING: yes
  Test for rank == 0, rank: 0
    hostname: "staff.frontera.tacc.utexas.edu" is rejected -> exiting
\}

Hello World!

myfini(0/1,LD\_PRELOAD,/home/user/t/xalt/hello)\{
  -> exiting because reject is set to: XALT has found the host does
  not match hostname pattern. If this is unexpected check your
  config.py file: Config/TACC\_config.py for program: /home/user/t/xalt/hello
\}
    \end{semiverbatim}
}
\end{frame}

% page 5
\begin{frame}[fragile]
    \frametitle{Tracing on compute node}
 {\tiny
    \begin{semiverbatim}
c202-001.fta[clx](3017)% XALT\_TRACING=run ./hello

myinit(0/1,LD\_PRELOAD,/home/user/t/xalt/hello)\{
  Test for \_\_XALT\_INITIAL\_STATE\_\_: "LD\_PRELOAD", STATE: "LD\_PRELOAD"
  Test for XALT\_EXECUTABLE\_TRACKING: yes
  Test for rank == 0, rank: 0
  GPU tracing is turned off. This directory "/sys/module/nvidia" does not exist!
  No GPU tracking
    -> Found watermark via vendor note: true
    -> MPI\_SIZE: 1 < MPI\_ALWAYS\_RECORD: 128, XALT is build to track
       all programs, Current program is a scalar
       program -> Not producing a start record
    -> Signals capturing disabled
    -> Leaving myinit
\}

Hello World!

myfini(0/1,LD\_PRELOAD,/home/user/t/xalt/hello)\{
    -> exiting because sampling. run\_time: 0.000298977, (my\_rand:
    0.943536 > prob: 0.0001) 
    for program: /home/user/t/xalt/hello
\}
   \end{semiverbatim}
}
\end{frame}

% page 6
\begin{frame}[fragile]
    \frametitle{Tracing on compute node, sampling off}
 {\tiny
    \begin{semiverbatim}
c202-001.fta[clx](3017)% XALT\_TRACING=run XALT\_SAMPLING=no ./hello

myinit(0/1,LD\_PRELOAD,/home/user/t/xalt/hello)\{
  ...
\}

Hello World!

myfini(0/1,LD\_PRELOAD,/home/user/t/xalt/hello)\{
    -> XALT\_SAMPLING = "no" All programs tracked!
  Recording State at end of scalar user program

  run\_submission(.zzz) \{
    Built processTree table
    Built envT
    Extracted recordT from watermark. Compiler: icc, XALT\_Version: 2.10.31
    Built userT, userDT, scheduler: SLURM
    Compute sha1 (...) of exec: /home/user/t/xalt/hello
    Parsed ProcMaps
    Using XALT\_TRANSMISSION\_STYLE: file
    Built json string
    Transmitting jsonStr
    Using USE\_HOME method to store json file
    Wrote json run file : ...
    Done transmitting jsonStr
  \}

    -> leaving myfini
\}
   \end{semiverbatim}
}
\end{frame}

% page 7
\begin{frame}[fragile]
    \frametitle{Tracing MPI program (I)}
 {\tiny
    \begin{semiverbatim}
\$ cat run_me.sh
XALT_MPI_ALWAYS_RECORD=2 XALT_TRACING=run XALT_SAMPLING=no ./mpi_hello
    \end{semiverbatim}
  }
\end{frame}

% page 8
\begin{frame}[fragile]
    \frametitle{Tracing MPI program (II)}
 {\tiny
    \begin{semiverbatim}
\$ ibrun -np 2 ./run\_me.sh
myinit(0/2,LD\_PRELOAD,/home/user/t/xalt/mpi\_hello)\{
  ...
    -> Found watermark via vendor note: true
    -> MPI\_SIZE: 2 >= MPI\_ALWAYS\_RECORD: 2 => recording start record!
  Recording state at beginning of MPI user program:
    /home/user/t/xalt/mpi\_hello

  run\_submission(.aaa) \{
    Built processTree table
    Built envT
    Extracted recordT from watermark. Compiler: mpicc(icc), XALT\_Version: 2.10.31
    Built userT, userDT, scheduler: SLURM
    Compute sha1 (...) of exec: /home/user/t/xalt/mpi\_hello
    Parsed ProcMaps
    Using XALT\_TRANSMISSION\_STYLE: file
    Built json string
    Transmitting jsonStr
    Using USE\_HOME method to store json file
    Wrote json run file : ...
    Done transmitting jsonStr
  \}
    -> uuid: b2f87517-8878-4449-bc93-5d899456b3c2
    -> Signals capturing disabled
    -> Leaving myinit
\}
Hello World from proc: 0 out of 2 !
...
   \end{semiverbatim}
}
\end{frame}

% page 9
\begin{frame}[fragile]
    \frametitle{Tracing MPI program (III)}
 {\tiny
    \begin{semiverbatim}
\$ ibrun -np 2 ./run\_me.sh
...
Hello World from proc: 0 out of 2 !

myfini(0/2,LD\_PRELOAD,/home/user/t/xalt/mpi\_hello){
  Recording State at end of MPI user program

  run\_submission(.zzz) {
    Built processTree table
    Built envT
    Extracted recordT from watermark. Compiler: mpicc(icc), XALT\_Version: 2.10.31
    Built userT, userDT, scheduler: SLURM
    Reuse   sha1 (...) of exec: /home/user/t/xalt/mpi\_hello
    Parsed ProcMaps
    Using XALT\_TRANSMISSION\_STYLE: file
    Built json string
    Transmitting jsonStr
    Using USE\_HOME method to store json file
    Wrote json run file : ...
    Done transmitting jsonStr
  }

    -> leaving myfini
}

TACC:  Shutdown complete. Exiting. 
\end{semiverbatim}
}
\end{frame}


% page 10
\begin{frame}{XALT Env. Vars for Tracing}
  \begin{itemize}
    \item XALT\_TRACING=run for execution
    \item XALT\_TRACING=link for linking
    \item XALT\_SAMPLING=no to turn off sampling
    \item XALT\_TESTING\_RUNTIME=3600 to override real runtime
    \item XALT\_MPI\_ALWAYS\_RECORD=2 to force start and end record generation.
    \item XALT\_COMM\_WORLD\_SIZE=256 to override number of real mpi tasks
      (XALT-2.10.33+)
  \end{itemize}
\end{frame}

%
%% page 6
%\begin{frame}{Full paths are remembered from 3 Env. Vars.}
%  \begin{itemize}
%    \item Changes to PATH
%    \item Changes to LD\\_LIBRARY\\_PATH
%    \item Setting $<$SITE$>$\\_$<$PKGName$>$\\_DIR
%    \item This last one is very important.
%  \end{itemize}
%\end{frame}
%

%\input{./themes/license}

\end{document}
