\documentclass{beamer}

% You can also use a 16:9 aspect ratio:
%\documentclass[aspectratio=169]{beamer}
\usetheme{TACC16}
\usepackage{graphicx}

% It's possible to move the footer to the right:
%\usetheme[rightfooter]{TACC16}


% Detailed knowledge of application workload characteristics can
% optimize performance of current and future systems. This may sound
% daunting, with many HPC data centers hosting over 2,000 users running
% thousands of applications and millions of jobs per month.  XALT is an
% open source tool developed at the Texas Advanced Computing Center
% (TACC) that collects system usage information to quantitatively report
% how users are using your system. This session will explore the
% benefits of detailed application workload profiling and how the XALT
% tool has helped leading supercomputing sites unlock the power of their
% application usage data.

%% page 
%\begin{frame}{}
%  \begin{itemize}
%    \item
%  \end{itemize}
%\end{frame}
%
%% page 
%\begin{frame}[fragile]
%    \frametitle{}
% {\tiny
%    \begin{semiverbatim}
%    \end{semiverbatim}
%}
%  \begin{itemize}
%    \item
%  \end{itemize}
%
%\end{frame}



\begin{document}
\title[XALT]{XALT 3.0: What Got Us Here}
\author{Robert McLay}
\date{June 15, 2023}

% page 1
\frame{\titlepage}

% page 2
\begin{frame}{XALT: Outline}
  \center{\includegraphics[width=.9\textwidth]{XALT_Sticker.png}}
  \begin{itemize}
    \item Releasing XALT 3.0 today
    \item Why change from XALT 2.10.*?
    \item What is new here XALT 3.0?
    \item A summary of XALT 2.0 and now
  \end{itemize}
\end{frame}

% page 3
\begin{frame}{XALT 3.0 release}
  \begin{itemize}
    \item I have been waiting for a good reason to bump XALT to 3.0
    \item It has been a long trip from the first release of XALT 2.0
    \item First xalt check-in was in 2014.
    \item The first release of XALT 2.0.0-devel was in April 17, 2018.
    \item XALT 3.0 is a evolution, not the revolution that 2.0 was.
  \end{itemize}
\end{frame}

% page 4
\begin{frame}{XALT 1.0 $\Rightfooter$ XALT 2.0}
  \begin{itemize}
    \item XALT 1.0 use mpirun/ibrun to capture runtime and info about
      mpi executions (Only MPI executions tracked)
    \item XALT 2.0 introduced the ELF trick to attach to \emph{every}
      execution to track all programs executions
    \item The journey to XALT 3.0 is how to deal with this fire hose of
      data
  \end{itemize}
\end{frame}

% page 5
\begin{frame}{Major changes to XALT since 2.0}
  \begin{itemize}
    \item Improved transport methods: SYSLOG, File, Curl
    \item Better debugging of XALT operations
    \item Filtering and Sampling
    \item Improved Performance
    \item Better Container Support.
    \item Support when UUID aren't
  \end{itemize}
\end{frame}

% page 6
\begin{frame}{Better Debugging of XALT Operations}
  \begin{itemize}
    \item Track what XALT does during execution
    \item XALT_TRACING=run try.exe
    \item Track what XALT does during Linking:
    \item XALT_TRACING=link gcc -o try.exe try.c
    \item Track what XALT does during ingestion:
    \item xalt_file_to_db.py -D ...
  \end{itemize}
\end{frame}

% page 7
\begin{frame}{XALT Filtering and Sampling}
  \begin{itemize}
    \item XALT generates a fire hose worth of data.
    \item Choice 1: Use big computer(s)  to collect the data
    \item Choice 2: Filter and Sample
    \item XALT can filter based on path
    \item (NEW) filtering based on command line arguments
    \item XALT does both filtering and sampling (Site configurable)
   \end{itemize}
\end{frame}

% page 8
\begin{frame}{XALT Sampling}
  \begin{itemize}
    \item Sampling based on execution time
    \item Different sampling rules possible for MPI and NON-MPI
      executions
    \item Next meeting: Design changes for Sampling and Signal
      handling support and failure
  \end{itemize}
\end{frame}

% page 9
\begin{frame}{Improved Performance}
  \begin{itemize}
    \item
  \end{itemize}
\end{frame}


% page 10
\begin{frame}{Conclusions}
  \begin{itemize}
    \item Available now in the testing branch of XALT
    \item It works with the cases I have tested with
    \item Some support for skipping over arguments.
    \item Please test it out.
  \end{itemize}
\end{frame}

% page 11
\begin{frame}{Future Topics?}
  \begin{itemize}
    \item Next Meeting will be on June15, 2023 at 10:00 am
      U.S. Central (15:00 UTC)
  \end{itemize}
\end{frame}

%\input{./themes/license}

\end{document}
