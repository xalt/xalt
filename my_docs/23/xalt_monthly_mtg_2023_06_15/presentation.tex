\documentclass{beamer}

% You can also use a 16:9 aspect ratio:
%\documentclass[aspectratio=169]{beamer}
\usetheme{TACC16}
\usepackage{graphicx}

% It's possible to move the footer to the right:
%\usetheme[rightfooter]{TACC16}


% Detailed knowledge of application workload characteristics can
% optimize performance of current and future systems. This may sound
% daunting, with many HPC data centers hosting over 2,000 users running
% thousands of applications and millions of jobs per month.  XALT is an
% open source tool developed at the Texas Advanced Computing Center
% (TACC) that collects system usage information to quantitatively report
% how users are using your system. This session will explore the
% benefits of detailed application workload profiling and how the XALT
% tool has helped leading supercomputing sites unlock the power of their
% application usage data.

%% page 
%\begin{frame}{}
%  \begin{itemize}
%    \item
%  \end{itemize}
%\end{frame}
%
%% page 
%\begin{frame}[fragile]
%    \frametitle{}
% {\tiny
%    \begin{semiverbatim}
%    \end{semiverbatim}
%}
%  \begin{itemize}
%    \item
%  \end{itemize}
%
%\end{frame}



\begin{document}
\title[XALT]{XALT 3.0: What Got Us Here}
\author{Robert McLay}
\date{June 15, 2023}

% page 1
\frame{\titlepage}

% page 2
\begin{frame}{XALT: Outline}
  \center{\includegraphics[width=.9\textwidth]{XALT_Sticker.png}}
  \begin{itemize}
    \item Releasing XALT 3.0 today
    \item Why change from XALT 2.10.*?
    \item What is new here XALT 3.0?
    \item A summary of changes from XALT 2.0 to now
  \end{itemize}
\end{frame}

% page 3
\begin{frame}{XALT 3.0 release}
  \begin{itemize}
    \item It has been a long trip from the first release of XALT 2.0
    \item First xalt check-in was in 2014.
    \item The first release of XALT 2.0.0-devel was in April 17, 2018.
    \item XALT 2.0 to 3.0 is a evolution, not the revolution that 1.0
      to 2.0 was.
  \end{itemize}
\end{frame}

% page 4
\begin{frame}{XALT 1.0 $\Rightarrow$ XALT 2.0}
  \begin{itemize}
    \item XALT 1.0 use mpirun/ibrun to capture runtime and info about
      mpi executions (Only MPI executions tracked)
    \item XALT 2.0 introduced the ELF trick to attach to \emph{every}
      execution to track all programs executions
    \item The journey to XALT 3.0 is how to deal with this fire hose of
      data
  \end{itemize}
\end{frame}

% page 5
\begin{frame}{Major changes to XALT since 2.0}
  \begin{itemize}
    \item Improved transport methods: SYSLOG, File, Curl
    \item Better debugging of XALT operations
    \item Filtering and Sampling
    \item Improved Performance
    \item Better Container Support.
    \item Support when UUID aren't
  \end{itemize}
\end{frame}

% page 6
\begin{frame}{Better Debugging of XALT Operations}
  \begin{itemize}
    \item Track what XALT does during execution
    \item XALT\_TRACING=run try.exe
    \item Track what XALT does during Linking:
    \item XALT\_TRACING=link gcc -o try.exe try.c
    \item Track what XALT does during ingestion:
    \item xalt\_file\_to\_db.py -D ...
  \end{itemize}
\end{frame}

% page 7
\begin{frame}{XALT Filtering and Sampling}
  \begin{itemize}
    \item XALT generates a fire hose worth of data.
    \item Choice 1: Use big computer(s)  to collect the data
    \item Choice 2: Filter and Sample
    \item XALT can filter based on path
    \item (NEW) filtering based on command line arguments
    \item XALT does both filtering and sampling (Site configurable)
   \end{itemize}
\end{frame}

% page 8
\begin{frame}{XALT Sampling}
  \begin{itemize}
    \item Sampling based on execution time
    \item Different sampling rules possible for MPI and NON-MPI
      executions
    \item Next meeting: Design changes for Sampling and Signal
      handling support and failure
  \end{itemize}
\end{frame}

% page 9
\begin{frame}{Improved Performance}
  \begin{itemize}
    \item Stopped taking SHA1 of shared libraries
    \item Full Tracking takes less than 0.01 seconds in my tests
    \item Timing are included in run.*.json files in the Q/A table.
  \end{itemize}
\end{frame}

% page 10
\begin{frame}{Better Container support}
  \begin{itemize}
    \item This has been the biggest area of change.
    \item All required libraries for XALT are copied to XALT
      installation directories.
    \item XALT uses dlopen to dynamically link in libuuid etc.
    \item There is support for Alpine containers (non-glibc based)
  \end{itemize}
\end{frame}

% page 11
\begin{frame}{Two support programs}
  \begin{itemize}
    \item \texttt{xalt\_configuration\_report}: How XALT is configured
      at your site
    \item \texttt{xalt\_extract\_record}: How, who and when an
      executable file was built.
  \end{itemize}
\end{frame}

% page 12
\begin{frame}{Protecing XALT from bad user programs}
  \begin{itemize}
    \item Better protection of malloc use in XALT
    \item XALT doesn't free memory to avoid user memory errors
  \end{itemize}
\end{frame}


% page 13
\begin{frame}{Some UUID implementations aren't }
  \begin{itemize}
    \item Older version of libuuid can produce duplicates
    \item XALT depends on UUID
    \item XALT now adds a CRC to the record to avoid issues when using
      Syslog transmission
  \end{itemize}
\end{frame}

% page 14
\begin{frame}{Separate C++ programs replaced for Run record}
  \begin{itemize}
    \item XALT used C based *.so to attach to each executable
    \item It used C++ programs to build json record (Hash Tables!!!)
    \item A version of MPI libraries prevented any execution after
      MPI\_finalize.
    \item This caused a re-write build the json run record in C (Not
      C++)
    \item I found a hash table implementation in C (uthash etc) (Not
      connnected to U Texas)
  \end{itemize}
\end{frame}

% page 15
\begin{frame}{Last Changes}
  \begin{itemize}
    \item Support for package filtering based on command line
      arguments
    \item Support for ARM based linux computers
    \item There is one piece of assembly code (the watermark!)
    \item Fortunately the gnu assembler is the same for X86 and ARM.
    \item This is in terms of how strings are stored.
  \end{itemize}
\end{frame}

% page 16
\begin{frame}{Conclusions}
  \begin{itemize}
    \item Many changes between XALT 2.0 and 3.0
    \item Very few bug reports in the last year.
  \end{itemize}
\end{frame}

% page 11
\begin{frame}{Future Topics?}
  \begin{itemize}
    \item Next Meeting will be on July 20, 2023 at 10:00 am
      U.S. Central (15:00 UTC)
    \item We will be discussing the design changes to support sampling
      and why signal won't save us.
  \end{itemize}
\end{frame}

%\input{./themes/license}

\end{document}
