\documentclass{beamer}

% You can also use a 16:9 aspect ratio:
%\documentclass[aspectratio=169]{beamer}
\usetheme{TACC16}
\usepackage{graphicx}

% It's possible to move the footer to the right:
%\usetheme[rightfooter]{TACC16}


% Detailed knowledge of application workload characteristics can
% optimize performance of current and future systems. This may sound
% daunting, with many HPC data centers hosting over 2,000 users running
% thousands of applications and millions of jobs per month.  XALT is an
% open source tool developed at the Texas Advanced Computing Center
% (TACC) that collects system usage information to quantitatively report
% how users are using your system. This session will explore the
% benefits of detailed application workload profiling and how the XALT
% tool has helped leading supercomputing sites unlock the power of their
% application usage data.

%% page 
%\begin{frame}{}
%  \begin{itemize}
%    \item
%  \end{itemize}
%\end{frame}
%
%% page 
%\begin{frame}[fragile]
%    \frametitle{}
% {\tiny
%    \begin{semiverbatim}
%    \end{semiverbatim}
%}
%  \begin{itemize}
%    \item
%  \end{itemize}
%
%\end{frame}



\begin{document}
\title[XALT]{How Sampling changed XALT and why Signals won't save us}
\author{Robert McLay}
\date{July 20, 2023}

% page 1
\frame{\titlepage}

% page 2
\begin{frame}{XALT: Outline}
  \center{\includegraphics[width=.9\textwidth]{XALT_Sticker.png}}
  \begin{itemize}
    \item XALT 2 can collect every execution
    \item This is too much data (at least for TACC)
    \item XALT 2 supports sampling to reduce the firehose of data.
    \item This brought many changes in the way XALT works
    \item I had hoped that signalling would allow XALT to drop the
      start record.  
  \end{itemize}
\end{frame}

% page 3
\begin{frame}{Too much data}
  \begin{itemize}
    \item One job used two nodes to generate 2 Million records
    \item It took over 4 days to load the 2 million records
    \item Then another job used small 2 node executions to train a
      neural network.
    \item We were drowning in data
  \end{itemize}
\end{frame}

% page 4
\begin{frame}{Too much data $\Rightarrow$ Sampling}
  \begin{itemize}
    \item Old XALT generated a start and end record for all executions
    \item This way failed executions could be tracked
    \item Site controlled sampling based on runtime.
    \item The longer the execution the more likely it will be
      ``tracked'' or recorded.
  \end{itemize}
\end{frame}

% page 5
\begin{frame}{Start Record?}
  \begin{itemize}
    \item XALT doesn't generate a start record for NON-MPI executions
    \item Do not want a start record that has to be ignored
    \item Problem: What about long running MPI programs that terminate
      by Job Scheduler
    \item No End Record
    \item Want to track these executions
  \end{itemize}
\end{frame}

% page 6
\begin{frame}{``Large'' MPI programs special treatment}
  \begin{itemize}
    \item Any execution MPI\_Tasks $<$ MPI\_ALWAYS\_RECORD (128 at TACC)
      will NOT generate a start record
    \item MPI\_Tasks $\geq$ MPI\_ALWAYS\_RECORD will generate a start
      record
    \item This way ``Big'' executions will always be tracked.
    \item If runtime $=$ zero then use job endtime to record runtime.
    \item This is an extra steps that sites must do
    \item This is outside of XALT data.
  \end{itemize}
\end{frame}

% page 7
\begin{frame}{Upshot of these rules}
  \begin{itemize}
    \item XALT knows nothing about non-MPI executions that fail
    \item XALT knows nothing about ``small'' MPI executions that fail
    \item XALT knows nothing about non-tracked execution
  \end{itemize}
\end{frame}

% page 8 
\begin{frame}{What about signals?}
  \begin{itemize}
    \item Use signals and get rid of the start record
      for ``Large'' MPI executions?
    \item SLURM sends a signal that a job is about to end.
    \item Can we use that?
  \end{itemize}
\end{frame}

% page 9
\begin{frame}{What about signals? (II)}
  \begin{itemize}
    \item Well, none of the MPI libraries I tested passed that signal through
      to the running program.
    \item Even if it did, XALT would require the signal that was
      raised on task zero.
  \end{itemize}
\end{frame}

% page 10
\begin{frame}{Conclusions}
  \begin{itemize}
    \item Sampling helps deal with the firehose of data
    \item But Signals won't save us from having to generate a start
      record on ``Large'' MPI executions.
  \end{itemize}
\end{frame}

% page 11
\begin{frame}{Future Topics?}
  \begin{itemize}
    \item Next Meeting will be on August 17, 2023 at 10:00 am
      U.S. Central (15:00 UTC)
  \end{itemize}
\end{frame}

%\input{./themes/license}

\end{document}
