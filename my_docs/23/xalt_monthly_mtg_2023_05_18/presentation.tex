\documentclass{beamer}

% You can also use a 16:9 aspect ratio:
%\documentclass[aspectratio=169]{beamer}
\usetheme{TACC16}
\usepackage{graphicx}

% It's possible to move the footer to the right:
%\usetheme[rightfooter]{TACC16}


% Detailed knowledge of application workload characteristics can
% optimize performance of current and future systems. This may sound
% daunting, with many HPC data centers hosting over 2,000 users running
% thousands of applications and millions of jobs per month.  XALT is an
% open source tool developed at the Texas Advanced Computing Center
% (TACC) that collects system usage information to quantitatively report
% how users are using your system. This session will explore the
% benefits of detailed application workload profiling and how the XALT
% tool has helped leading supercomputing sites unlock the power of their
% application usage data.

%% page 
%\begin{frame}{}
%  \begin{itemize}
%    \item
%  \end{itemize}
%\end{frame}
%
%% page 
%\begin{frame}[fragile]
%    \frametitle{}
% {\tiny
%    \begin{semiverbatim}
%    \end{semiverbatim}
%}
%  \begin{itemize}
%    \item
%  \end{itemize}
%
%\end{frame}



\begin{document}
\title[XALT]{Path arg updates}
\author{Robert McLay}
\date{May 18, 2023}

% page 1
\frame{\titlepage}

% page 2
\begin{frame}{XALT: Outline}
  \center{\includegraphics[width=.9\textwidth]{XALT_Sticker.png}}
  \begin{itemize}
    \item This is an update from the last meeting in March.
    \item Wrong before: Python3 has two options that take arguments
    \item \texttt{python3 -c "print(3.0/2)"}
    \item \texttt{python3 -m "json.tool" run.json}
    \item It is currently available under the testing branch
  \end{itemize}
\end{frame}

% page 3
\begin{frame}{Reminder: XALT does the filtering for you}
  \begin{itemize}
    \item This consists of two parts in your site Config.py file
    \item Changes to path\_patterns
    \item Add in a new group of patterns: path\_arg\_patterns
  \end{itemize}
\end{frame}

% page 4
\begin{frame}[fragile]
    \frametitle{Adding the CUSTOM tag to path\_patterns}
 {\tiny
    \begin{semiverbatim}
path_patterns = [
    ['CUSTOM',  r'.*\\/python[0-9][^/][^/]*'],
    ...
]
    \end{semiverbatim}
}
  \begin{itemize}
    \item Only paths that have the ``CUSTOM'' tag will get further filtering
    \item A site can have as many ``CUSTOM'' tags as they like
  \end{itemize}

\end{frame}

% page 5
\begin{frame}[fragile]
    \frametitle{The new \texttt{path\_arg\_pattern}}
 {\tiny
    \begin{semiverbatim}
path_arg_patterns = [
  ['JUMP_1',r'.*\\/python[0-9][^/;][^/;]*;-m'],
  ['SKIP',  r'.*\/python[0-9][^/;][^/;]*;-c'],
  ['SKIP',  r'.*\\/python[0-9][^/;][^/;]*;.*\\/share\\/.*'],
  ['PKGS',  r'.*\\/python[0-9][^/;][^/;]*;.*\\/data\\/.*'],
  ['PKGS',  r'.*\/python[0-9][^/;][^/;]*;'],
]
    \end{semiverbatim}
}
  \begin{itemize}
    \item The pattern is path + ``;'' + arg as shown above
      \begin{enumerate}
        \item SKIP any python scripts that have /share/ as part of the
          path
        \item KEEP any python scripts that have /data/ as part of the
        \item KEEP any python scripts that have neither of the above
      \end{enumerate}
    \item Note the change in how the executable pattern is written!
    \item The first python line skips over -m option and continues parsing
    \item The second python line with -c says to ignore this command.
  \end{itemize}

\end{frame}

% page 6
\begin{frame}[fragile]
    \frametitle{Rules}
 {\tiny
    \begin{semiverbatim}
    \end{semiverbatim}
}
  \begin{itemize}
    \item For ``CUSTOM'' tags the arguments are each processed
    \item Arguments that start with minus [-] are ignored
    \item All other arguments would be abspath and checked for existance
      before being run through path\_arg\_patterns
    \item Not perfect but reasonably safe
  \end{itemize}
\end{frame}


% page 7
\begin{frame}{Show this in action}
  \begin{itemize}
    \item xalt\_configuration\_report output
    \item example\_run.txt
  \end{itemize}
\end{frame}

% page 8
\begin{frame}{Other news}
  \begin{itemize}
    \item Testing XALT on M1 Mac with UTM running linux
    \item Problem was that XALT writes an assembly watermark
    \item Converted \# comments to /* comments */
    \item This updated xalt.s now builds on Arm based linux
    \item I will test XALT on other ARM platforms in the next month
  \end{itemize}
\end{frame}

% page 9
\begin{frame}{Other news (II)}
  \begin{itemize}
    \item Once path arg testing is complete
    \item And XALT now runs on several ARM platforms
    \item The next version of XALT will be 3.0
  \end{itemize}
\end{frame}

% page 10
\begin{frame}{Conclusions}
  \begin{itemize}
    \item Available now in the testing branch of XALT
    \item It works with the cases I have tested with
    \item Some support for skipping over arguments.
    \item Please test it out.
  \end{itemize}
\end{frame}

% page 11
\begin{frame}{Future Topics?}
  \begin{itemize}
    \item Next Meeting will be on June15, 2023 at 10:00 am
      U.S. Central (15:00 UTC)
  \end{itemize}
\end{frame}

%\input{./themes/license}

\end{document}
